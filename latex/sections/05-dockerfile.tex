% 05-dockerfile.tex
% Dockerfile セクション

\chapter{Dockerfile と OCR 環境}

\section{概要}

このセクションでは、Python + Tesseract OCR を備えた本番環境対応の Dockerfile を実装します。

% ================================================================================
% コマンド定義
% ================================================================================
\newcommand{\cmd}[1]{\texttt{#1}}  % コマンド表示用
\newcommand{\tip}[1]{\textbf{ヒント:} #1}  % ヒント表示用

\section{Dockerfile 実装}

\begin{lstlisting}[language=dockerfile,caption=本番対応 Dockerfile]
# Dockerfile
# Copilot 完全自動化ガイド - OCR 環境
# マルチステージビルド対応
# 作成日: 2025-12-10

# ========================================
# ステージ 1: ビルドステージ
# ========================================
FROM python:3.11-slim as builder

LABEL maintainer="Copilot Automation <info@example.com>"
LABEL description="OCR Processing Environment for Copilot Automation"
LABEL version="1.0.0"

# 作業ディレクトリ設定
WORKDIR /build

# ビルド依存関係インストール
RUN apt-get update && apt-get install -y --no-install-recommends \
    build-essential \
    gcc \
    g++ \
    git \
    && rm -rf /var/lib/apt/lists/*

# Python 依存関係ダウンロード
COPY requirements.txt .
RUN pip install --no-cache-dir --upgrade pip setuptools wheel && \
    pip download --no-cache-dir \
    --dest /pip-packages \
    -r requirements.txt

# ========================================
# ステージ 2: ランタイムステージ
# ========================================
FROM python:3.11-slim

# メタデータ
ARG BUILD_DATE
ARG VCS_REF
LABEL org.label-schema.build-date=$BUILD_DATE
LABEL org.label-schema.vcs-ref=$VCS_REF

# 非ルートユーザー作成
RUN groupadd -r appuser && useradd -r -g appuser appuser

# 作業ディレクトリ
WORKDIR /app

# システムパッケージをインストール(OCR)
RUN apt-get update && apt-get install -y --no-install-recommends \
    tesseract-ocr \
    tesseract-ocr-jpn \
    tesseract-ocr-eng \
    libtesseract-dev \
    leptonica-progs \
    libmagic1 \
    fonts-noto-cjk \
    && rm -rf /var/lib/apt/lists/*

# キャッシュマウント経由で効率的にインストール
RUN --mount=type=cache,target=/root/.cache/pip \
    pip install --upgrade pip setuptools wheel && \
    pip install --no-cache-dir \
    pytesseract==0.3.10 \
    pillow==10.0.0 \
    opencv-python==4.8.0.76 \
    numpy==1.24.3 \
    pandas==2.0.3 \
    pydantic==2.0.0

# 依存ファイル コピー(本番)
RUN --mount=type=cache,target=/root/.cache/pip \
    pip install --no-cache-dir -r requirements.txt || true

# アプリケーション コピー
COPY --chown=appuser:appuser . .

# ディレクトリパーミッション設定
RUN mkdir -p /app/input /app/output && \
    chown -R appuser:appuser /app && \
    chmod 755 /app/input /app/output

# 非ルートユーザーに切り替え
USER appuser

# ヘルスチェック
HEALTHCHECK --interval=30s --timeout=10s --start-period=5s --retries=3 \
    CMD python -c "import pytesseract; pytesseract.get_tesseract_version()" || exit 1

# デフォルトコマンド
ENTRYPOINT ["python"]
CMD ["main.py"]

# ========================================
# ステージ 3: テストステージ(オプション)
# ========================================
FROM python:3.11-slim as test

WORKDIR /app

# テストツール
RUN apt-get update && apt-get install -y --no-install-recommends \
    pytest \
    pytest-cov \
    && rm -rf /var/lib/apt/lists/*

COPY --from=builder /pip-packages /pip-packages
RUN pip install /pip-packages/* && \
    pip install pytest pytest-cov

COPY . .

# テスト実行
CMD ["pytest", "-v", "--cov=.", "tests/"]
\end{lstlisting}

\section{requirements.txt}

\begin{lstlisting}[caption=requirements.txt - Python 依存関係]
# requirements.txt
# Python OCR 環境 依存関係
# Python 3.9+

# === OCR & 画像処理 ===
pytesseract==0.3.10          # Tesseract Python インターフェース
pillow==10.0.0               # 画像処理ライブラリ
opencv-python==4.8.0.76      # コンピュータビジョン
numpy==1.24.3                # 数値計算

# === データ処理 ===
pandas==2.0.3                # データ分析
openpyxl==3.1.2             # Excel 処理
pydantic==2.0.0              # データバリデーション

# === ファイルハンドリング ===
python-magic-bin==0.4.14    # ファイル型認識

# === ロギング & モニタリング ===
python-json-logger==2.0.7   # JSON ロギング
prometheus-client==0.17.1   # Prometheus メトリクス

# === Web & API ===
requests==2.31.0            # HTTP ライブラリ
httpx==0.24.1               # 非同期 HTTP

# === 設定管理 ===
python-dotenv==1.0.0        # .env ファイル読み込み

# === デバッグ & 開発 ===
pytest==7.4.0               # テストフレームワーク(開発用)
pytest-cov==4.1.0           # カバレッジ測定
black==23.7.0               # コードフォーマッター
flake8==6.0.0               # リンター

# === セキュリティ ===
cryptography==41.0.3        # 暗号化
\end{lstlisting}

\section{Docker 実行方法}

\subsection{ローカルビルド}

\begin{lstlisting}[language=bash,caption=Docker イメージビルド]
# イメージビルド
docker build -t copilot-automation:latest .

# タグ付け(GHCR 用)
docker tag copilot-automation:latest \
  ghcr.io/your-org/copilot-automation:latest
\end{lstlisting}

\subsection{コンテナ実行(OCR 処理例)}

\begin{lstlisting}[language=bash,caption=Docker コンテナ実行]
# 入出力ボリュームマウント
docker run --rm \
  --name copilot-ocr \
  -v $(pwd)/input:/app/input \
  -v $(pwd)/output:/app/output \
  copilot-automation:latest \
  main.py

# 環境変数指定
docker run --rm \
  -e LOG_LEVEL=DEBUG \
  -e OCR_LANGUAGE=jpn+eng \
  -v $(pwd)/input:/app/input \
  -v $(pwd)/output:/app/output \
  copilot-automation:latest
\end{lstlisting}

\subsection{Docker Compose(開発環境)}

\begin{lstlisting}[language=yaml,caption=docker-compose.yml]
version: '3.8'

services:
  ocr-processor:
    build:
      context: .
      dockerfile: Dockerfile
    container_name: copilot-ocr
    environment:
      - LOG_LEVEL=INFO
      - OCR_LANGUAGE=jpn+eng
      - TZ=Asia/Tokyo
    volumes:
      - ./input:/app/input
      - ./output:/app/output
      - ./logs:/app/logs
    networks:
      - copilot-net
    restart: on-failure

  # オプション: nginx リバースプロキシ
  nginx:
    image: nginx:alpine
    ports:
      - "80:80"
    volumes:
      - ./output:/usr/share/nginx/html:ro
    depends_on:
      - ocr-processor
    networks:
      - copilot-net

networks:
  copilot-net:
    driver: bridge
\end{lstlisting}

\section{main.py - OCR スクリプト例}

\begin{lstlisting}[language=python,caption=main.py - OCR メイン処理]
#!/usr/bin/env python3
# -*- coding: utf-8 -*-
"""
main.py - OCR 処理メインスクリプト
Tesseract OCR を使用してPDF/画像からテキストを抽出し CSV に出力
"""

import os
import sys
import logging
from pathlib import Path
import pytesseract
from PIL import Image
import pandas as pd

# ロギング設定
logging.basicConfig(
    level=os.getenv('LOG_LEVEL', 'INFO'),
    format='%(asctime)s - %(name)s - %(levelname)s - %(message)s'
)
logger = logging.getLogger(__name__)

# 入出力パス
INPUT_DIR = Path('/app/input')
OUTPUT_DIR = Path('/app/output')

def extract_text_from_image(image_path: Path) -> str:
    """
    画像ファイルから OCR でテキスト抽出
    
    Args:
        image_path: 画像ファイルパス
    
    Returns:
        抽出されたテキスト
    """
    try:
        image = Image.open(image_path)
        # 日本語 + 英語で OCR
        text = pytesseract.image_to_string(
            image,
            lang='jpn+eng'
        )
        logger.info(f"✓ {image_path.name} から OCR 抽出完了")
        return text
    except Exception as e:
        logger.error(f"✗ OCR 抽出失敗 {image_path}: {e}")
        return ""

def process_input_directory():
    """入力ディレクトリ内の全ファイルを処理"""
    results = []
    
    # サポート形式
    image_extensions = {'.jpg', '.jpeg', '.png', '.tif', '.tiff'}
    
    for file_path in INPUT_DIR.glob('*'):
        if file_path.suffix.lower() not in image_extensions:
            logger.warning(f"スキップ: {file_path.name} (非対応形式)")
            continue
        
        logger.info(f"処理中: {file_path.name}")
        text = extract_text_from_image(file_path)
        
        results.append({
            'filename': file_path.name,
            'extracted_text': text,
            'text_length': len(text)
        })
    
    return results

def save_results_to_csv(results: list):
    """
    結果を CSV に保存
    
    Args:
        results: OCR 結果リスト
    """
    if not results:
        logger.warning("処理対象ファイルがありません")
        return
    
    df = pd.DataFrame(results)
    output_file = OUTPUT_DIR / 'ocr_results.csv'
    
    try:
        df.to_csv(output_file, index=False, encoding='utf-8-sig')
        logger.info(f"✓ CSV 保存成功: {output_file}")
        logger.info(f"  処理ファイル数: {len(results)}")
    except Exception as e:
        logger.error(f"✗ CSV 保存失敗: {e}")

def main():
    """メイン処理"""
    logger.info("=" * 50)
    logger.info("OCR 処理スクリプト開始")
    logger.info("=" * 50)
    
    # ディレクトリ確認
    if not INPUT_DIR.exists():
        logger.error(f"入力ディレクトリが見つかりません: {INPUT_DIR}")
        return 1
    
    OUTPUT_DIR.mkdir(exist_ok=True)
    
    # 処理実行
    results = process_input_directory()
    save_results_to_csv(results)
    
    logger.info("=" * 50)
    logger.info("OCR 処理完了")
    logger.info("=" * 50)
    return 0

if __name__ == '__main__':
    sys.exit(main())
\end{lstlisting}

\section{セキュリティベストプラクティス}

\begin{itemize}
  \item \textbf{マルチステージビルド}: 本番イメージサイズ最小化
  \item \textbf{非ルートユーザー}: \cmd{USER appuser} で権限制限
  \item \textbf{ヘルスチェック}: コンテナ健全性自動監視
  \item \textbf{キャッシュマウント}: ビルド高速化
  \item \textbf{.dockerignore}: センシティブファイル除外
\end{itemize}

\tip{イメージサイズは約 1.2GB(Tesseract + 日本語フォント込み)です。
本番環境では Amazon ECR や Docker Hub キャッシュを活用してください。}
