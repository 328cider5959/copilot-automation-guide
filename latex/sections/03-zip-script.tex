% 03-zip-script.tex
% ZIP スクリプトセクション

\chapter{ZIP スクリプトと自動パッケージング}

\section{概要}

ZIP スクリプトは \file{starter-kit/} フォルダを圧縮し、GitHub Release にアップロード可能な
パッケージを生成します。

\section{bash スクリプト実装}

\begin{lstlisting}[language=bash,caption=generate-starter-kit.sh - ZIP生成スクリプト]
#!/bin/bash
# generate-starter-kit.sh
# スターターキットを ZIP ファイルにパッケージング
# 作成日: 2025-12-10
# 対応: bash 4.0+, Windows (WSL/Git Bash)

set -euo pipefail

# ===================================
# 設定
# ===================================
SCRIPT_DIR="$(cd "$(dirname "${BASH_SOURCE[0]}")" && pwd)"
PROJECT_ROOT="$(dirname "$SCRIPT_DIR")"
STARTER_KIT_DIR="$PROJECT_ROOT/starter-kit"
OUTPUT_DIR="$PROJECT_ROOT/output"
TIMESTAMP=$(date +%Y%m%d_%H%M%S)
ZIP_NAME="starter-kit_${TIMESTAMP}.zip"
ZIP_PATH="$OUTPUT_DIR/$ZIP_NAME"

# ===================================
# 色付き出力
# ===================================
RED='\033[0;31m'
GREEN='\033[0;32m'
YELLOW='\033[1;33m'
BLUE='\033[0;34m'
NC='\033[0m' # No Color

# ===================================
# ログ関数
# ===================================
log_info() {
    echo -e "${BLUE}[INFO]${NC} $1"
}

log_success() {
    echo -e "${GREEN}[SUCCESS]${NC} $1"
}

log_error() {
    echo -e "${RED}[ERROR]${NC} $1" >&2
}

log_warning() {
    echo -e "${YELLOW}[WARNING]${NC} $1"
}

# ===================================
# 前提条件チェック
# ===================================
check_prerequisites() {
    log_info "前提条件をチェック中..."
    
    # zip コマンド確認
    if ! command -v zip &> /dev/null; then
        log_error "zip コマンドが見つかりません。インストールしてください。"
        exit 1
    fi
    
    # スターターキット ディレクトリ確認
    if [ ! -d "$STARTER_KIT_DIR" ]; then
        log_error "スターターキット ディレクトリ $STARTER_KIT_DIR が見つかりません。"
        exit 1
    fi
    
    log_success "前提条件チェック完了"
}

# ===================================
# 出力ディレクトリ作成
# ===================================
prepare_output_dir() {
    log_info "出力ディレクトリを準備中..."
    
    if [ ! -d "$OUTPUT_DIR" ]; then
        mkdir -p "$OUTPUT_DIR"
        log_info "出力ディレクトリを作成: $OUTPUT_DIR"
    fi
    
    log_success "出力ディレクトリ準備完了"
}

# ===================================
# ファイル検証
# ===================================
validate_files() {
    log_info "スターターキット内のファイルを検証中..."
    
    local required_files=(
        "README.md"
        "config"
        "templates"
        "scripts"
        "docs"
        "examples"
    )
    
    for file in "${required_files[@]}"; do
        if [ ! -e "$STARTER_KIT_DIR/$file" ]; then
            log_warning "推奨ファイル/ディレクトリが見つかりません: $file"
        else
            log_info "✓ $file"
        fi
    done
    
    log_success "ファイル検証完了"
}

# ===================================
# ZIP ファイル生成
# ===================================
create_zip() {
    log_info "ZIP ファイルを生成中..."
    log_info "対象ディレクトリ: $STARTER_KIT_DIR"
    log_info "出力先: $ZIP_PATH"
    
    # 既存の ZIP を削除
    if [ -f "$ZIP_PATH" ]; then
        log_warning "既存 ZIP を削除: $ZIP_PATH"
        rm -f "$ZIP_PATH"
    fi
    
    # ZIP 生成
    cd "$PROJECT_ROOT"
    zip -r -q "$ZIP_PATH" "starter-kit/" \
        -x "starter-kit/.DS_Store" \
           "starter-kit/*/.DS_Store" \
           "starter-kit/__pycache__/*" \
           "starter-kit/*.pyc"
    
    if [ -f "$ZIP_PATH" ]; then
        log_success "ZIP ファイル生成成功: $ZIP_PATH"
        
        # ファイルサイズ表示
        local size=$(du -h "$ZIP_PATH" | cut -f1)
        log_info "ファイルサイズ: $size"
    else
        log_error "ZIP ファイル生成に失敗しました。"
        exit 1
    fi
}

# ===================================
# ZIP 内容確認
# ===================================
verify_zip() {
    log_info "ZIP ファイルの内容を検証中..."
    
    local file_count=$(unzip -l "$ZIP_PATH" | tail -1 | awk '{print $2}')
    log_info "ZIP に含まれるファイル数: $file_count"
    
    # 重要ファイル確認
    if unzip -l "$ZIP_PATH" | grep -q "README.md"; then
        log_success "✓ README.md が含まれています"
    else
        log_warning "⚠ README.md が見つかりません"
    fi
    
    log_success "ZIP 検証完了"
}

# ===================================
# チェックサム生成
# ===================================
generate_checksum() {
    log_info "チェックサムを生成中..."
    
    local checksum_file="$OUTPUT_DIR/${ZIP_NAME%.zip}.sha256"
    
    if command -v sha256sum &> /dev/null; then
        sha256sum "$ZIP_PATH" > "$checksum_file"
        log_success "SHA256 チェックサム生成: $checksum_file"
    elif command -v shasum &> /dev/null; then
        shasum -a 256 "$ZIP_PATH" > "$checksum_file"
        log_success "SHA256 チェックサム生成: $checksum_file"
    else
        log_warning "チェックサム生成ツールが見つかりません。スキップします。"
    fi
}

# ===================================
# メイン処理
# ===================================
main() {
    log_info "========================================"
    log_info "  Starter Kit ZIP 生成スクリプト"
    log_info "========================================"
    
    check_prerequisites
    prepare_output_dir
    validate_files
    create_zip
    verify_zip
    generate_checksum
    
    log_info "========================================"
    log_success "スターターキットの ZIP 生成が完了しました!"
    log_info "出力: $ZIP_PATH"
    log_info "========================================"
}

# エラートラップ
trap 'log_error "スクリプト実行中にエラーが発生しました"; exit 1' ERR

# メイン実行
main "$@"
\end{lstlisting}

\section{スクリプト実行方法}

\subsection{Linux/macOS}

\begin{lstlisting}[language=bash,caption=Linux/macOS での実行]
chmod +x scripts/generate-starter-kit.sh
bash scripts/generate-starter-kit.sh
\end{lstlisting}

\subsection{Windows (PowerShell)}

\begin{lstlisting}[language=bash,caption=Windows PowerShell での実行]
# WSL 環境がある場合
wsl bash scripts/generate-starter-kit.sh

# または Git Bash で実行
bash scripts/generate-starter-kit.sh
\end{lstlisting}

\section{出力ファイル}

スクリプト実行後、以下のファイルが \file{output/} に生成されます:

\begin{table}[h]
\centering
\begin{tabularx}{\textwidth}{|l|X|}
\hline
\textbf{ファイル} & \textbf{説明} \\
\hline
\file{starter-kit\_YYYYMMDD\_HHMMSS.zip} & スターターキット ZIP \\
\file{starter-kit\_YYYYMMDD\_HHMMSS.sha256} & SHA256 チェックサム \\
\hline
\end{tabularx}
\caption{生成されるファイル}
\end{table}

\section{ベストプラクティス}

\begin{itemize}
  \item \textbf{定期実行}: CI/CD パイプラインで自動実行
  \item \textbf{チェックサム}: 整合性確認に SHA256 チェックサムを使用
  \item \textbf{ログ確認}: エラー時は詳細ログで原因を特定
  \item \textbf{バージョン管理}: ZIP ファイルは Git LFS で管理
\end{itemize}

\note{Windows 環境では、ZIP 生成前に 7-Zip や WinRAR などのツールで
テストすることをお勧めします。}
