% 08-manual.tex
% 実装手順セクション

\chapter{実装手順書}

\section{全体フロー}

本セクションでは、ゼロからプロジェクトを立ち上げるための
ステップバイステップガイドを提供します。

\subsection{前提条件確認}

\begin{lstlisting}[language=bash,caption=前提条件チェック]
#!/bin/bash

echo "=== 前提条件チェック ==="

# Git
git --version || { echo "✗ Git をインストールしてください"; exit 1; }

# Docker
docker --version || { echo "✗ Docker をインストールしてください"; exit 1; }

# Python
python3 --version || { echo "✗ Python 3.9+ をインストールしてください"; exit 1; }

# TeX Live (オプション)
pdflatex --version && echo "✓ LaTeX 環境あり" || echo "⚠ LaTeX なし (PDF 生成スキップ)"

echo "✓ 前提条件チェック完了"
\end{lstlisting}

\section{ステップ 1: GitHub リポジトリセットアップ}

\subsection{1-1: リポジトリ作成}

\begin{enumerate}
  \item GitHub にログイン: \url{https://github.com}
  \item \cmd{New} → リポジトリ作成
  \item リポジトリ名: \textit{copilot-automation-guide}
  \item 説明: \textit{Copilot Complete Automation Guide}
  \item Visibility: \textit{Public} (または Private)
  \item \cmd{Create repository}
\end{enumerate}

\subsection{1-2: リポジトリクローン}

\begin{lstlisting}[language=bash,caption=リポジトリクローン]
git clone https://github.com/YOUR-USERNAME/copilot-automation-guide.git
cd copilot-automation-guide

# ブランチ確認
git branch -a

# develop ブランチ作成(推奨)
git checkout -b develop
\end{lstlisting}

\subsection{1-3: 初期ファイル配置}

\begin{lstlisting}[language=bash,caption=ファイル配置]
# リポジトリルート
touch README.md LICENSE .gitignore

# ディレクトリ作成
mkdir -p .github/workflows
mkdir -p latex/sections
mkdir -p starter-kit/{config,templates,scripts,docs,examples}
mkdir -p power-automate
mkdir -p scripts

# 本ガイドから各ファイルをコピー
\end{lstlisting}

\section{ステップ 2: GitHub Secrets 設定}

\subsection{2-1: Personal Access Token (PAT) 生成}

\begin{enumerate}
  \item GitHub Settings → \cmd{Developer settings} → \cmd{Personal access tokens}
  \item \cmd{Generate new token (classic)}
  \item Token 名: \textit{GHCR_DEPLOYMENT_PAT}
  \item スコープ: \checkbox{} \cmd{repo}, \checkbox{} \cmd{write:packages}
  \item \cmd{Generate token}
  \item トークンをコピーして安全に保存
\end{enumerate}

\subsection{2-2: Secrets 設定}

\begin{enumerate}
  \item リポジトリ Settings → \cmd{Secrets and variables} → \cmd{Actions}
  \item \cmd{New repository secret}
  \item 以下を追加:
\end{enumerate}

\begin{table}[h]
\centering
\begin{tabularx}{\textwidth}{|l|X|}
\hline
\textbf{シークレット名} & \textbf{値} \\
\hline
\cmd{PAT\_TOKEN} & 上記で生成した PAT トークン \\
\cmd{POWER\_AUTOMATE\_WEBHOOK} & Power Automate Webhook URL \\
\cmd{GITHUB\_TOKEN} & 自動(リポジトリシークレット) \\
\hline
\end{tabularx}
\caption{GitHub Secrets 設定}
\end{table}

\section{ステップ 3: ローカル開発環境構築}

\subsection{3-1: Python 仮想環境}

\begin{lstlisting}[language=bash,caption=Python 仮想環境構築]
# 仮想環境作成
python3 -m venv venv

# アクティベート
source venv/bin/activate  # Linux/macOS
# または
.\venv\Scripts\activate  # Windows

# 依存関係インストール
pip install --upgrade pip
pip install -r requirements.txt
\end{lstlisting}

\subsection{3-2: Docker ローカルテスト}

\begin{lstlisting}[language=bash,caption=Docker イメージローカルビルド]
# イメージビルド
docker build -t copilot-automation:dev .

# コンテナ実行テスト
mkdir -p input output
echo "テスト画像" > input/test.txt

docker run --rm \
  -v $(pwd)/input:/app/input \
  -v $(pwd)/output:/app/output \
  copilot-automation:dev
\end{lstlisting}

\section{ステップ 4: ワークフロー実行}

\subsection{4-1: GitHub Actions ワークフロー トリガー}

\begin{enumerate}
  \item リポジトリ → \cmd{Actions}
  \item \cmd{Build and Deploy Pipeline} ワークフロー選択
  \item \cmd{Run workflow} → \cmd{Run workflow}
\end{enumerate}

\subsection{4-2: 実行状況確認}

\begin{lstlisting}[language=bash,caption=ワークフロー ログ確認]
# コマンドラインから確認(gh CLI)
gh run list
gh run view <RUN_ID> --log
\end{lstlisting}

\section{ステップ 5: タグリリース}

\subsection{5-1: バージョンタグ作成}

\begin{lstlisting}[language=bash,caption=リリースタグ作成]
# バージョンタグ作成
git tag -a v1.0.0 -m "Release version 1.0.0"

# リモートプッシュ
git push origin v1.0.0

# タグ確認
git tag -l
\end{lstlisting}

\subsection{5-2: GitHub Release 確認}

\begin{enumerate}
  \item リポジトリ → \cmd{Releases}
  \item v1.0.0 をクリック
  \item PDF と Starter Kit ZIP が自動アップロードされていることを確認
\end{enumerate}

\section{ステップ 6: Power Automate 接続}

\subsection{6-1: Webhook URL 生成}

\begin{enumerate}
  \item Power Automate → \cmd{Create} → \cmd{Instant cloud flow}
  \item \cmd{Request} トリガー選択
  \item \cmd{Save}
  \item トリガー内の Webhook URL をコピー
\end{enumerate}

\subsection{6-2: GitHub Secret に登録}

\begin{bash}
gh secret set POWER_AUTOMATE_WEBHOOK --body "https://prod-xx.westus.logic.azure.com:443/..."
\end{bash}

\section{ステップ 7: 検証とテスト}

\subsection{7-1: End-to-End テスト}

\begin{lstlisting}[language=bash,caption=E2E テストシーケンス]
#!/bin/bash

echo "=== End-to-End Test Suite ==="

# 1. LaTeX PDF 生成テスト
echo "[1/5] Testing LaTeX PDF generation..."
cd latex
pdflatex -interaction=nonstopmode guide.tex
pdflatex -interaction=nonstopmode guide.tex
if [ -f guide.pdf ]; then echo "✓ PDF OK"; else echo "✗ PDF NG"; fi

# 2. Docker ビルドテスト
echo "[2/5] Testing Docker build..."
cd ..
docker build -t copilot-automation:test . && echo "✓ Docker OK" || echo "✗ Docker NG"

# 3. ZIP 生成テスト
echo "[3/5] Testing ZIP generation..."
bash scripts/generate-starter-kit.sh && echo "✓ ZIP OK" || echo "✗ ZIP NG"

# 4. Python 依存テスト
echo "[4/5] Testing Python dependencies..."
python -m pip check && echo "✓ Dependencies OK" || echo "✗ Dependencies NG"

# 5. GitHub Secrets テスト
echo "[5/5] Testing GitHub Secrets..."
[ -n "$POWER_AUTOMATE_WEBHOOK" ] && echo "✓ Secrets OK" || echo "✗ Secrets NG"

echo "=== Test Complete ==="
\end{lstlisting}

\subsection{7-2: トラブルシューティング}

\begin{table}[h]
\centering
\begin{tabularx}{\textwidth}{|l|X|l|}
\hline
\textbf{問題} & \textbf{原因} & \textbf{解決策} \\
\hline
LaTeX エラー & フォント不足 & \cmd{tlmgr install cjk} \\
Docker エラー & イメージ大きい & キャッシュクリア \\
Secret 未設定 & リポジトリ設定ミス & Settings 再確認 \\
\hline
\end{tabularx}
\caption{よくあるトラブルと対応}
\end{table}

\section{ステップ 8: 本番環境デプロイ}

\subsection{8-1: ブランチ保護設定}

\begin{enumerate}
  \item Settings → \cmd{Branches} → \cmd{Branch protection rules}
  \item \cmd{Add rule}
  \item Branch name pattern: \textit{main}
  \item \checkbox{} Require status checks
  \item \checkbox{} Require reviews before merging
\end{enumerate}

\subsection{8-2: CI/CD パイプライン検証}

\begin{itemize}
  \item全 Job が成功することを確認
  \item アーティファクトが正常にアップロードされていることを確認
  \item Release ページで PDF と ZIP がダウンロード可能か確認
\end{itemize}

\note{本番環境では必ず develop ブランチで十分なテストを行い、
その後 main ブランチにマージしてください。}
