% 02-starter-kit.tex
% スターターキット構成セクション

\chapter{スターターキット構成}

\section{スターターキットの概要}

スターターキット (\file{starter-kit/}) は、プロジェクト開始時に必要な全テンプレート、
設定ファイル、ドキュメントをまとめたパッケージです。

\subsection{目的}

\begin{itemize}
  \item 新しいプロジェクト立ち上げの高速化
  \item 標準的なファイル構造の提供
  \item ベストプラクティスの実装例
  \item セキュリティ設定の自動化
\end{itemize}

\section{ディレクトリ構造}

\begin{lstlisting}[language=bash,caption=スターターキットの構成]
starter-kit/
├── README.md                      # スターターキット ガイド
├── CONTRIBUTING.md                # 貢献ガイドライン
│
├── config/
│   ├── github-actions.yaml        # GitHub Actions テンプレート
│   ├── docker-compose.yml         # Docker Compose 設定
│   ├── .env.example               # 環境変数テンプレート
│   └── security-policy.md         # セキュリティポリシー
│
├── templates/
│   ├── issue-template.md          # Issue テンプレート
│   ├── pr-template.md             # Pull Request テンプレート
│   ├── dockerfile.template        # Dockerfile テンプレート
│   ├── workflow.yaml.template     # ワークフロー YAML テンプレート
│   └── python-main.py.template    # Python スクリプト テンプレート
│
├── scripts/
│   ├── setup.sh                   # 初期セットアップ スクリプト
│   ├── validate.sh                # 設定検証スクリプト
│   ├── deploy.sh                  # デプロイスクリプト
│   └── README.md                  # スクリプト ドキュメント
│
├── docs/
│   ├── GETTING_STARTED.md         # 始め方ガイド
│   ├── ARCHITECTURE.md            # アーキテクチャ説明
│   ├── DEPLOYMENT.md              # デプロイ手順
│   ├── TROUBLESHOOTING.md         # トラブルシューティング
│   └── FAQ.md                     # よくある質問
│
└── examples/
    ├── python/
    │   ├── main.py
    │   └── requirements.txt
    ├── bash/
    │   ├── example.sh
    │   └── utilities.sh
    └── yaml/
        ├── workflow-example.yaml
        └── docker-compose-example.yml
\end{lstlisting}

\section{各ファイルの説明}

\subsection{設定ファイル (config/)}

\begin{description}
  \item[\file{github-actions.yaml}] GitHub Actions ワークフロー設定テンプレート
  \item[\file{docker-compose.yml}] ローカル開発環境用 Docker Compose
  \item[\file{.env.example}] 環境変数テンプレート(本番では \file{.env} に改名)
  \item[\file{security-policy.md}] 脆弱性報告ポリシー
\end{description}

\subsection{テンプレート (templates/)}

\begin{lstlisting}[caption=Issue テンプレート例]
<!-- issue-template.md -->
## 問題の説明
<!-- 何が起きたか簡潔に説明 -->

## 再現手順
1. ステップ 1
2. ステップ 2
3. ステップ 3

## 期待される動作
<!-- 何が起こるべきだったか -->

## 実際の動作
<!-- 実際に起きたこと -->

## スクリーンショット(オプション)
<!-- スクリーンショットを添付 -->

## 環境情報
- OS: 
- Docker Version: 
- Python Version: 
\end{lstlisting}

\subsection{スクリプト (scripts/)}

\begin{table}[h]
\centering
\begin{tabularx}{\textwidth}{|l|X|}
\hline
\textbf{スクリプト} & \textbf{機能} \\
\hline
\file{setup.sh} & 環境の初期セットアップ(Python venv、依存関係インストール等) \\
\file{validate.sh} & 設定ファイル、環境変数、依存関係の検証 \\
\file{deploy.sh} & Docker イメージビルドと GHCR へのプッシュ \\
\hline
\end{tabularx}
\caption{スターターキット スクリプト}
\end{table}

\section{使用開始手順}

\subsection{ステップ 1: ZIP から展開}

\begin{lstlisting}[language=bash,caption=スターターキット展開]
unzip starter-kit.zip
cd starter-kit
\end{lstlisting}

\subsection{ステップ 2: 初期セットアップ実行}

\begin{lstlisting}[language=bash,caption=セットアップスクリプト実行]
bash scripts/setup.sh
\end{lstlisting}

\subsection{ステップ 3: 環境変数設定}

\begin{lstlisting}[language=bash,caption=環境変数ファイル作成]
cp config/.env.example .env
# .env をテキストエディタで編集して、実際の値を設定
\end{lstlisting}

\subsection{ステップ 4: 設定検証}

\begin{lstlisting}[language=bash,caption=設定検証実行]
bash scripts/validate.sh
\end{lstlisting}

\section{カスタマイズガイド}

\subsection{独自の GitHub Actions ワークフロー作成}

\begin{lstlisting}[language=yaml,caption=カスタム ワークフロー例]
name: Custom Build Pipeline

on:
  push:
    branches: [ main ]

jobs:
  build:
    runs-on: ubuntu-latest
    steps:
      - uses: actions/checkout@v3
      - name: Set up Python
        uses: actions/setup-python@v4
        with:
          python-version: '3.11'
      - name: Run tests
        run: python -m pytest
\end{lstlisting}

\subsection{Docker イメージのカスタマイズ}

\begin{lstlisting}[language=dockerfile,caption=カスタム Dockerfile]
FROM python:3.11-slim

# 作業ディレクトリ設定
WORKDIR /app

# 依存関係インストール
COPY requirements.txt .
RUN pip install --no-cache-dir -r requirements.txt

# アプリケーション コピー
COPY . .

# デフォルト コマンド
CMD ["python", "main.py"]
\end{lstlisting}

\section{ベストプラクティス}

\begin{itemize}
  \item \textbf{バージョン管理}: スターターキットも Git で管理する
  \item \textbf{ドキュメント}: 変更時は対応するドキュメントも更新
  \item \textbf{テスト}: 新規スクリプト追加時は自動テストを含める
  \item \textbf{セキュリティ}: センシティブ情報は環境変数に移行
\end{itemize}

\tip{スターターキットは GitHub Templates リポジトリとしても利用できます。}
