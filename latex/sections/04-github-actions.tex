% 04-github-actions.tex
% GitHub Actions セクション

\chapter{GitHub Actions YAML ワークフロー}

\section{概要}

GitHub Actions ワークフローは、リポジトリへのプッシュやリリース作成時に
自動的に実行される CI/CD パイプラインです。

\section{完全な build.yml 実装}

\begin{lstlisting}[language=yaml,caption=.github/workflows/build.yml - 完全版]
name: Build and Deploy Pipeline

# トリガー条件
on:
  push:
    branches:
      - main
      - develop
    tags:
      - 'v*'
  pull_request:
    branches:
      - main
  workflow_dispatch:  # 手動トリガー

# 環境変数(グローバル)
env:
  REGISTRY: ghcr.io
  IMAGE_NAME: ${{ github.repository }}
  LATEX_VERSION: texlive-2024

# 並列ジョブ実行
jobs:
  # ========================================
  # Job 1: LaTeX PDF ビルド
  # ========================================
  build-latex:
    name: Build LaTeX PDF
    runs-on: ubuntu-latest
    outputs:
      pdf_artifact: ${{ steps.build.outputs.pdf_name }}
    
    steps:
      # チェックアウト
      - name: Checkout repository
        uses: actions/checkout@v4
      
      # LaTeX 環境キャッシュ
      - name: Cache TeX Live
        uses: actions/cache@v3
        with:
          path: ~/.texlive2024/texmf-var
          key: ${{ runner.os }}-texlive-${{ env.LATEX_VERSION }}-${{ hashFiles('**/preamble.tex') }}
          restore-keys: |
            ${{ runner.os }}-texlive-${{ env.LATEX_VERSION }}-
      
      # LaTeX インストール
      - name: Install LaTeX dependencies
        run: |
          sudo apt-get update
          sudo apt-get install -y texlive-full texlive-xetex texlive-lang-japanese
          sudo apt-get install -y fonts-noto-cjk
      
      # PDF ビルド(2 回実行:目次生成用)
      - name: Build LaTeX PDF
        id: build
        run: |
          cd latex
          pdflatex -interaction=nonstopmode -halt-on-error guide.tex
          pdflatex -interaction=nonstopmode -halt-on-error guide.tex
          echo "pdf_name=guide.pdf" >> $GITHUB_OUTPUT
          ls -lah *.pdf
      
      # ビルド成功確認
      - name: Verify PDF generation
        run: |
          if [ -f "latex/guide.pdf" ]; then
            echo "✓ PDF ビルド成功"
            file latex/guide.pdf
          else
            echo "✗ PDF ビルド失敗"
            exit 1
          fi
      
      # PDF をアーティファクトとして保存
      - name: Upload PDF artifact
        uses: actions/upload-artifact@v3
        with:
          name: latex-pdf
          path: latex/guide.pdf
          retention-days: 30

  # ========================================
  # Job 2: Docker イメージビルド
  # ========================================
  build-docker:
    name: Build Docker Image
    runs-on: ubuntu-latest
    permissions:
      contents: read
      packages: write
    
    steps:
      - name: Checkout repository
        uses: actions/checkout@v4
      
      # Docker メタデータ生成
      - name: Extract metadata
        id: meta
        uses: docker/metadata-action@v5
        with:
          images: ${{ env.REGISTRY }}/${{ env.IMAGE_NAME }}
          tags: |
            type=semver,pattern={{version}}
            type=semver,pattern={{major}}.{{minor}}
            type=ref,event=branch
            type=sha,prefix={{branch}}-
      
      # Docker buildx セットアップ
      - name: Set up Docker Buildx
        uses: docker/setup-buildx-action@v2
      
      # GHCR ログイン
      - name: Log in to Container Registry
        uses: docker/login-action@v2
        with:
          registry: ${{ env.REGISTRY }}
          username: ${{ github.actor }}
          password: ${{ secrets.GITHUB_TOKEN }}
      
      # Docker イメージビルド & プッシュ
      - name: Build and push Docker image
        uses: docker/build-push-action@v4
        with:
          context: .
          push: true
          tags: ${{ steps.meta.outputs.tags }}
          labels: ${{ steps.meta.outputs.labels }}
          cache-from: type=gha
          cache-to: type=gha,mode=max
          build-args: |
            BUILD_DATE=${{ github.event.head_commit.timestamp }}
            VCS_REF=${{ github.sha }}

  # ========================================
  # Job 3: スターターキット生成
  # ========================================
  create-starter-kit:
    name: Create Starter Kit ZIP
    runs-on: ubuntu-latest
    outputs:
      zip_path: ${{ steps.create.outputs.zip_path }}
    
    steps:
      - name: Checkout repository
        uses: actions/checkout@v4
      
      - name: Create ZIP file
        id: create
        run: |
          OUTPUT_DIR="output"
          mkdir -p "$OUTPUT_DIR"
          
          ZIP_NAME="starter-kit_${{ github.run_id }}.zip"
          ZIP_PATH="$OUTPUT_DIR/$ZIP_NAME"
          
          cd ..
          zip -r -q "${{ github.workspace }}/$ZIP_PATH" \
            "copilot-automation-guide/starter-kit/" \
            -x "*/.*" "*/__pycache__/*" "*.pyc"
          
          echo "zip_path=$ZIP_PATH" >> $GITHUB_OUTPUT
          ls -lah "$ZIP_PATH"
      
      - name: Upload ZIP artifact
        uses: actions/upload-artifact@v3
        with:
          name: starter-kit
          path: ${{ steps.create.outputs.zip_path }}

  # ========================================
  # Job 4: GitHub Release にアップロード
  # ========================================
  upload-release:
    name: Upload to GitHub Release
    runs-on: ubuntu-latest
    needs: [build-latex, create-starter-kit]
    if: startsWith(github.ref, 'refs/tags/')
    
    steps:
      - name: Checkout repository
        uses: actions/checkout@v4
      
      - name: Download PDF artifact
        uses: actions/download-artifact@v3
        with:
          name: latex-pdf
          path: release-files
      
      - name: Download Starter Kit artifact
        uses: actions/download-artifact@v3
        with:
          name: starter-kit
          path: release-files
      
      - name: Create Release
        uses: softprops/action-gh-release@v1
        with:
          files: release-files/*
          draft: false
          prerelease: false
          generate_release_notes: true
        env:
          GITHUB_TOKEN: ${{ secrets.GITHUB_TOKEN }}

  # ========================================
  # Job 5: Power Automate トリガー
  # ========================================
  trigger-power-automate:
    name: Trigger Power Automate Flow
    runs-on: ubuntu-latest
    needs: upload-release
    if: startsWith(github.ref, 'refs/tags/')
    
    steps:
      - name: Get Release Info
        id: release
        run: |
          RELEASE_TAG=${{ github.ref }}
          RELEASE_TAG=${RELEASE_TAG#refs/tags/}
          echo "tag=$RELEASE_TAG" >> $GITHUB_OUTPUT
      
      - name: Trigger Power Automate Webhook
        run: |
          curl -X POST \
            "${{ secrets.POWER_AUTOMATE_WEBHOOK }}" \
            -H "Content-Type: application/json" \
            -d '{
              "tag": "${{ steps.release.outputs.tag }}",
              "repository": "${{ github.repository }}",
              "releaseUrl": "${{ github.server_url }}/${{ github.repository }}/releases/tag/${{ steps.release.outputs.tag }}",
              "triggeredBy": "${{ github.actor }}"
            }'
        env:
          GITHUB_TOKEN: ${{ secrets.GITHUB_TOKEN }}

  # ========================================
  # Job 6: テストとバリデーション
  # ========================================
  validate:
    name: Validate Build Artifacts
    runs-on: ubuntu-latest
    needs: [build-latex, build-docker, create-starter-kit]
    
    steps:
      - name: Download all artifacts
        uses: actions/download-artifact@v3
      
      - name: List artifacts
        run: |
          echo "=== Build Artifacts ===" 
          find . -type f -name "*.pdf" -o -name "*.zip" | sort
      
      - name: Verify file integrity
        run: |
          if [ ! -f "latex-pdf/guide.pdf" ]; then
            echo "✗ PDF が見つかりません"
            exit 1
          fi
          if [ ! -f "starter-kit/"*.zip ]; then
            echo "✗ Starter Kit ZIP が見つかりません"
            exit 1
          fi
          echo "✓ すべてのアーティファクト確認完了"

# ========================================
# ワークフロー終了後の処理
# ========================================
on-workflow-end:
  name: Cleanup and Notify
  runs-on: ubuntu-latest
  if: always()
  needs: [build-latex, build-docker, create-starter-kit, validate]
  
  steps:
    - name: Send Slack Notification
      if: failure()
      uses: slackapi/slack-github-action@v1
      with:
        webhook-url: ${{ secrets.SLACK_WEBHOOK }}
        payload: |
          {
            "text": "Build Pipeline Failed",
            "details": "${{ github.server_url }}/${{ github.repository }}/actions/runs/${{ github.run_id }}"
          }
\end{lstlisting}

\section{ワークフロー実行フロー}

\begin{figure}[h]
\centering
\begin{verbatim}
Trigger Event
    │
    ├─→ build-latex (並列)
    │   ├─ Cache TeX Live
    │   ├─ Install Dependencies
    │   ├─ pdflatex (2x)
    │   └─ Upload Artifact
    │
    ├─→ build-docker (並列)
    │   ├─ Extract Metadata
    │   ├─ Setup Buildx
    │   ├─ Login to Registry
    │   └─ Build & Push Image
    │
    └─→ create-starter-kit (並列)
        ├─ Create ZIP
        └─ Upload Artifact
         
    All Jobs Complete
         │
         ├─→ validate (依存)
         │   ├─ Download Artifacts
         │   └─ Verify Integrity
         │
         └─→ upload-release (条件: Tag Push)
             ├─ Create GitHub Release
             └─ Upload Files
                  │
                  └─→ trigger-power-automate
                      └─ Call Webhook
\end{verbatim}
\caption{ワークフロー実行シーケンス}
\end{figure}

\section{シークレット設定}

GitHub Settings から以下のシークレットを設定:

\begin{table}[h]
\centering
\begin{tabularx}{\textwidth}{|l|X|l|}
\hline
\textbf{名前} & \textbf{値} & \textbf{優先度} \\
\hline
\cmd{GITHUB\_TOKEN} & 自動設定 & ⭐⭐⭐ \\
\cmd{POWER\_AUTOMATE\_WEBHOOK} & Webhook URL & ⭐⭐⭐ \\
\cmd{SLACK\_WEBHOOK} & Slack Webhook (オプション) & ⭐⭐ \\
\hline
\end{tabularx}
\caption{必要なシークレット}
\end{table}

\section{トラブルシューティング}

\subsection{LaTeX ビルド失敗}

\begin{lstlisting}[language=bash,caption=デバッグ対策]
# ローカルで pdflatex テスト
cd latex
pdflatex -interaction=nonstopmode -halt-on-error guide.tex 2>&1 | tail -20
\end{lstlisting}

\subsection{Docker ビルド失敗}

原因の可能性:
\begin{itemize}
  \item キャッシュレイヤー問題 → \cmd{cache-to: type=gha,mode=max} を確認
  \item レジストリログイン → \cmd{GITHUB\_TOKEN} 権限を確認
  \item ディスク容量 → ランナーの空き容量を確認
\end{itemize}

\tip{GitHub Actions ランナーは 14GB メモリ、160GB SSD を備えています。}
