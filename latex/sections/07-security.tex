% 07-security.tex
% セキュリティセクション

\chapter{セキュリティベストプラクティス}

\section{概要}

本番環境における CI/CD パイプライン、コンテナ、クラウドサービスの
セキュリティは最優先です。本セクションでは実装済みのセキュリティ対策を詳述します。

\section{シークレット管理}

\subsection{GitHub Secrets}

\begin{lstlisting}[language=yaml,caption=シークレット参照(安全)]
# build.yml での使用方法

# ✓ 正しい方法:シークレットから参照
env:
  GITHUB_TOKEN: ${{ secrets.GITHUB_TOKEN }}
  PAT_TOKEN: ${{ secrets.PAT_TOKEN }}

jobs:
  deploy:
    steps:
      - name: Deploy
        env:
          # ✓ ステップレベルでシークレット参照
          DOCKER_PASSWORD: ${{ secrets.DOCKER_PASSWORD }}
        run: |
          # ✗ ログ出力禁止
          # echo $DOCKER_PASSWORD
          
          # ✓ マスク処理
          echo "::add-mask::$DOCKER_PASSWORD"
\end{lstlisting}

\subsection{環境変数とシークレット}

\begin{table}[h]
\centering
\begin{tabularx}{\textwidth}{|l|X|X|}
\hline
\textbf{タイプ} & \textbf{用途} & \textbf{管理方法} \\
\hline
シークレット & パスワード、トークン、API キー & GitHub Secrets \\
環境変数 & ビルド設定、ログレベル & .env ファイル \\
設定ファイル & アプリケーション設定 & config/ ディレクトリ \\
\hline
\end{tabularx}
\caption{情報管理の分類}
\end{table}

\section{Docker セキュリティ}

\subsection{マルチステージビルド}

\begin{lstlisting}[language=dockerfile,caption=セキュアな マルチステージビルド]
# ステージ 1: ビルド(開発ツール含む)
FROM python:3.11-slim as builder

WORKDIR /build

# ビルド依存(本番では不要)
RUN apt-get update && apt-get install -y gcc g++ make

COPY requirements.txt .
RUN pip install -r requirements.txt

# ステージ 2: ランタイム(最小限)
FROM python:3.11-slim

# 非ルートユーザー作成
RUN groupadd -r appuser && useradd -r -g appuser appuser

COPY --from=builder /usr/local/lib/python3.11/site-packages \
  /usr/local/lib/python3.11/site-packages

USER appuser  # 非ルートユーザーに切り替え

CMD ["python", "main.py"]
\end{lstlisting}

\subsection{イメージスキャンとセキュリティチェック}

\begin{lstlisting}[language=yaml,caption=trivy によるスキャン]
- name: Scan Docker Image with Trivy
  uses: aquasecurity/trivy-action@master
  with:
    image-ref: 'copilot-automation:latest'
    format: 'sarif'
    output: 'trivy-results.sarif'

- name: Upload Trivy results
  uses: github/codeql-action/upload-sarif@v2
  with:
    sarif_file: 'trivy-results.sarif'
\end{lstlisting}

\section{ネットワーク セキュリティ}

\subsection{HTTPS 強制}

\begin{lstlisting}[language=yaml,caption=HTTPS トラフィック強制]
jobs:
  deploy:
    steps:
      - name: Verify HTTPS
        run: |
          # URL スキーム検証
          if [[ ! "$WEBHOOK_URL" =~ ^https:// ]]; then
            echo "✗ HTTPS が必須です"
            exit 1
          fi
          echo "✓ HTTPS 確認完了"
        env:
          WEBHOOK_URL: ${{ secrets.POWER_AUTOMATE_WEBHOOK }}
\end{lstlisting}

\subsection{API キーローテーション}

\begin{itemize}
  \item \textbf{定期実行}: 30-90 日ごと
  \item \textbf{ローテーション方法}: 新キー追加 → 古キー削除 → 再デプロイ
  \item \textbf{自動化}: GitHub Actions スケジュール実行
  \item \textbf{監査}: 全ローテーション操作をログ記録
\end{itemize}

\section{コード セキュリティ}

\subsection{依存関係チェック}

\begin{lstlisting}[language=yaml,caption=Dependabot による脆弱性検出]
# .github/dependabot.yml
version: 2
updates:
  # Python 依存関係
  - package-ecosystem: "pip"
    directory: "/"
    schedule:
      interval: "daily"
    security-updates-only: true
    reviewers:
      - "security-team"

  # Docker ベースイメージ
  - package-ecosystem: "docker"
    directory: "/"
    schedule:
      interval: "weekly"

  # GitHub Actions
  - package-ecosystem: "github-actions"
    directory: "/"
    schedule:
      interval: "weekly"
\end{lstlisting}

\subsection{SAST (Static Application Security Testing)}

\begin{lstlisting}[language=yaml,caption=CodeQL による脆弱性解析]
- name: Initialize CodeQL
  uses: github/codeql-action/init@v2
  with:
    languages: 'python'

- name: Run CodeQL analysis
  uses: github/codeql-action/analyze@v2

- name: Upload CodeQL results
  uses: github/codeql-action/upload-sarif@v2
  with:
    sarif_file: 'codeql-results.sarif'
\end{lstlisting}

\section{ログと監査}

\subsection{セキュリティログ記録}

\begin{lstlisting}[language=python,caption=Python セキュリティログ]
import logging
import json
from datetime import datetime

# セキュアなロギング設定
logging.basicConfig(
    level=logging.INFO,
    format='%(asctime)s - %(name)s - %(levelname)s - %(message)s',
    handlers=[
        logging.FileHandler('security.log'),
        logging.StreamHandler()
    ]
)

logger = logging.getLogger(__name__)

def log_security_event(event_type, user, action, result):
    """セキュリティイベント ロギング"""
    log_entry = {
        'timestamp': datetime.utcnow().isoformat(),
        'event_type': event_type,
        'user': user,
        'action': action,
        'result': result
    }
    logger.info(json.dumps(log_entry))

# 使用例
log_security_event(
    event_type='deployment',
    user='github-actions',
    action='push_docker_image',
    result='success'
)
\end{lstlisting}

\subsection{監査トレイル}

\begin{itemize}
  \item \textbf{GitHub}: Settings → Audit log で全操作を監視
  \item \textbf{Docker}: GHCR → Package settings で push/pull ログを確認
  \item \textbf{OneDrive}: SharePoint → Site contents で ファイル変更追跡
  \item \textbf{Power Automate}: Run history で フロー実行履歴確認
\end{itemize}

\section{リスク評価と対応}

\subsection{セキュリティリスク行列}

\begin{table}[h]
\centering
\begin{tabularx}{\textwidth}{|l|l|l|l|}
\hline
\textbf{リスク} & \textbf{影響} & \textbf{可能性} & \textbf{対応} \\
\hline
シークレット漏洩 & 🔴 致命的 & 中 & GitHub Secrets \\
脆弱性 in deps & 🟠 高 & 高 & Dependabot \\
無許可アクセス & 🔴 致命的 & 低 & 2FA, OIDC \\
DoS 攻撃 & 🟡 中 & 低 & レート制限 \\
\hline
\end{tabularx}
\caption{セキュリティリスク評価}
\end{table}

\section{コンプライアンス}

\subsection{チェックリスト}

\begin{itemize}
  \item \checkbox{} GitHub Secret で全シークレット管理
  \item \checkbox{} Docker イメージ定期スキャン (Trivy)
  \item \checkbox{} Dependabot 有効化
  \item \checkbox{} CodeQL 分析実行
  \item \checkbox{} セキュリティログ記録
  \item \checkbox{} 監査トレイル確認
  \item \checkbox{} API キー 90 日ローテーション
  \item \checkbox{} HTTPS のみ使用
  \item \checkbox{} 非ルートユーザー実行
  \item \checkbox{} パーミッション最小化 (RBAC)
\end{itemize}

\note{本番環境では全項目をチェック状態にしてください。}
