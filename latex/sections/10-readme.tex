% 10-readme.tex
% README 内容セクション

\chapter{README 内容と応用}

\section{README.md の役割}

README.md はプロジェクトの入口となるドキュメントです。
本セクションでは、効果的な README 構成と実装方法を解説します。

\section{README 最適構成}

\begin{lstlisting}[language=markdown,caption=README.md の最適構成テンプレート]
# プロジェクト名

[![Badge1](url)](link)
[![Badge2](url)](link)

## 🎯 概要

1 段落で何ができるかを説明(最大 100 語)

### 主な特徴

- ✓ 特徴 1
- ✓ 特徴 2
- ✓ 特徴 3

## 📋 目次

- [インストール](#インストール)
- [使い方](#使い方)
- [アーキテクチャ](#アーキテクチャ)
- [貢献](#貢献)

## 🚀 クイックスタート

\`\`\`bash
# 最小限のセットアップコマンド
git clone https://github.com/...
cd project
bash setup.sh
\`\`\`

## 📦 インストール

### 前提条件
- Git 2.20+
- Docker 20.10+

### インストール手順

1. リポジトリクローン
2. 環境変数設定
3. 依存関係インストール

### シークレット設定

| Key | Value |
|-----|-------|
| GITHUB_TOKEN | auto |
| PAT_TOKEN | your-pat |

## 💻 使い方

### 基本使用例

\`\`\`bash
# ビルド
docker build -t myapp:latest .

# 実行
docker run myapp:latest
\`\`\`

### 詳細な使用例

[詳細ドキュメント](docs/USAGE.md) を参照

## 🏗️ アーキテクチャ

[図を表示]

## 🔒 セキュリティ

[セキュリティ ポリシー](SECURITY.md)

## 📚 ドキュメント

- [ビルド手順](docs/BUILD.md)
- [デプロイ手順](docs/DEPLOY.md)
- [トラブルシューティング](docs/TROUBLESHOOTING.md)

## 🤝 貢献

プルリクエスト歓迎!
[貢献ガイドライン](CONTRIBUTING.md)

## 📄 ライセンス

[MIT License](LICENSE)

## 👥 お問い合わせ

- 📧 Email: info@example.com
- 🐦 Twitter: @handle
- 💬 Discussions: GitHub Discussions
\end{lstlisting}

\section{バッジの追加方法}

\subsection{一般的なバッジ}

\begin{table}[h]
\centering
\begin{tabularx}{\textwidth}{|l|X|}
\hline
\textbf{バッジ} & \textbf{Markdown} \\
\hline
Build Status & \verb|[![Build](url/badge.svg)](url)| \\
License & \verb|[![MIT](url/badge.svg)](LICENSE)| \\
Python Version & \verb|[![Python 3.11](url/badge.svg)](url)| \\
Docker & \verb|[![Docker](url/badge.svg)](url)| \\
\hline
\end{tabularx}
\caption{一般的なバッジ}
\end{table}

\subsection{バッジ生成ツール}

\begin{itemize}
  \item \textbf{Shields.io}: \url{https://shields.io}
  \item \textbf{GitHub Actions Status}: 自動生成
  \item \textbf{Codecov}: カバレッジバッジ
  \item \textbf{Docker Hub}: イメージメタデータ
\end{itemize}

\section{表の効果的な使用}

\begin{lstlisting}[language=markdown,caption=README 内の表の例]
## 機能比較表

| 機能 | 無料版 | Pro版 | Enterprise |
|------|--------|-------|-----------|
| 基本機能 | ✓ | ✓ | ✓ |
| CI/CD | - | ✓ | ✓ |
| カスタムドメイン | - | ✓ | ✓ |
| SLA保証 | - | - | ✓ |

## 環境別ワークフロー表

| 環境 | 用途 | トリガー | 出力 |
|------|------|---------|------|
| dev | 開発テスト | Push to develop | ログ出力 |
| staging | 本番前テスト | PR to main | Test report |
| production | 本番運用 | Release tag | Deployment |
\end{lstlisting}

\section{コード例の効果的な提示}

\subsection{言語指定}

\begin{lstlisting}[language=markdown,caption=複数言語の code block]
# Python 例
\`\`\`python
import os
print(os.getenv('MY_VAR'))
\`\`\`

# bash 例
\`\`\`bash
export MY_VAR=value
\`\`\`

# Docker コマンド例
\`\`\`docker
FROM python:3.11-slim
RUN pip install requests
\`\`\`
\end{lstlisting}

\section{セクション別 README テンプレート}

\subsection{新規プロジェクト用テンプレート}

\begin{lstlisting}[language=markdown,caption=新規プロジェクト README]
# My Awesome Project

## 概要
このプロジェクトは [説明] を目指しています。

## インストール

\`\`\`bash
git clone ...
cd ...
pip install -r requirements.txt
\`\`\`

## 使い方

\`\`\`bash
python main.py --help
\`\`\`

## ライセンス
MIT

## 貢献
Pull Request 歓迎!
\end{lstlisting}

\subsection{既存プロジェクト更新用テンプレート}

\begin{lstlisting}[language=markdown,caption=更新内容を記載した README]
## 変更履歴

### v2.0.0 (2025-12-10)
- ✨ 新機能 A を追加
- 🐛 バグ B を修正
- ⚡ パフォーマンス C を改善

### v1.5.0 (2025-11-10)
- ...

## マイグレーション ガイド

v1.x から v2.0 への更新方法:

1. 依存関係を更新: \`pip install --upgrade package\`
2. 設定ファイルを移行: [マイグレーション手順](docs/MIGRATE.md)
3. テストを再実行
\end{lstlisting}

\section{よくある質問 (FAQ) セクション}

\begin{lstlisting}[language=markdown,caption=README に FAQ を追加]
## よくある質問

### Q: インストールに失敗した
A: [トラブルシューティング](docs/TROUBLESHOOTING.md) を確認してください。

### Q: Docker イメージはどこにある?
A: [GHCR Registry](https://ghcr.io/your-org/project) で公開しています。

### Q: ライセンスは?
A: [MIT License](LICENSE) です。商用利用可能です。

### Q: 日本語ドキュメントはありますか?
A: [日本語ドキュメント](docs/README_ja.md) を参照してください。
\end{lstlisting}

\section{README チェックリスト}

プロジェクト公開前に確認:

\begin{itemize}
  \item \checkbox{} 明確なプロジェクト説明
  \item \checkbox{} インストール手順
  \item \checkbox{} 使用例(コード付き)
  \item \checkbox{} ライセンス明記
  \item \checkbox{} 貢献ガイドリンク
  \item \checkbox{} トラブルシューティング
  \item \checkbox{} FAQ
  \item \checkbox{} バッジ付き
  \item \checkbox{} リンク確認(全て有効)
  \item \checkbox{} スペリングチェック(特に日本語)
  \item \checkbox{} コード例が実行可能か確認
  \item \checkbox{} ディレクトリ構造が最新か確認
  \item \checkbox{} 前提条件の明記
  \item \checkbox{} 必要なシークレット情報の記載
  \item \checkbox{} サポート窓口情報の記載
\end{itemize}

\section{README メンテナンス}

\subsection{定期更新スケジュール}

\begin{table}[h]
\centering
\begin{tabularx}{\textwidth}{|l|l|}
\hline
\textbf{頻度} & \textbf{更新内容} \\
\hline
毎月 & バージョン情報、リリースノート \\
四半期 & バッジ更新、依存関係の最小バージョン \\
必要に応じて & バグ修正、新機能追加、FAQ 追加 \\
\hline
\end{tabularx}
\caption{README メンテナンススケジュール}
\end{table}

\subsection{バージョン管理}

\begin{lstlisting}[language=bash,caption=README バージョン管理]
# README 内にバージョン情報を含める
---
version: 2.0.0
last-updated: 2025-12-10
maintained: true
---

# README をコミット時に日付付けで記録
git log --oneline -- README.md | head -10
\end{lstlisting}

\section{多言語対応 README}

\begin{lstlisting}[language=markdown,caption=多言語 README サポート]
# プロジェクト名

[日本語](README_ja.md) | [English](README.md) | [中文](README_zh.md)

## English Version

# Project Name

[Content in English]

---

## 日本語版

[日本語コンテンツ]
\end{lstlisting}

\note{GitHub は複数言語の README.md を自動検出します。
言語別ファイル名: README\_en.md, README\_ja.md, etc.}

\section{README 自動生成ツール}

\begin{itemize}
  \item \textbf{readme-md-generator}: npm パッケージ
  \item \textbf{auto-readme}: Python ツール
  \item \textbf{GitHub CLI}: \cmd{gh api} で自動化
\end{itemize}

例:

\begin{lstlisting}[language=bash,caption=readme-md-generator 使用例]
npm install -g readme-md-generator

# 対話的に README 生成
readme
\end{lstlisting}
