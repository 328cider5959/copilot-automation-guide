%!TEX encoding = UTF-8 Unicode
% -*- coding: utf-8 -*-
% Copilot 完全自動化ガイド 最終版
% LaTeX Ultimate PDF Version

\documentclass[11pt,a4paper,dvipdfmx,uplatex]{jreport}

% ========================================
% パッケージ読み込み
% ========================================

% 日本語フォント設定
\usepackage[utf8]{inputenc}
\usepackage[T1]{fontenc}
\usepackage{lmodern}

% CJK フォント(日本語対応)
\usepackage[dvipdfmx]{graphicx}
\usepackage{xcolor}
\usepackage{amsmath,amssymb}

% ハイパーリンク
\usepackage[dvipdfmx,bookmarks=true,colorlinks=true,
            linkcolor=blue,urlcolor=blue,citecolor=blue]{hyperref}

% コード表示(listings環境)
\usepackage{listings}
\usepackage{jlisting}  % 日本語対応listings

% フォント設定
\usepackage{xltxtra}
\usepackage{xunicode}
\defaultfontfeatures{Mapping=tex-text}

% レイアウト
\usepackage[margin=25mm,headheight=15pt]{geometry}
\usepackage{fancyhdr}
\usepackage{lastpage}

% 目次の自動生成
\usepackage{titletoc}
\usepackage{tocloft}

% テーブル
\usepackage{booktabs}
\usepackage{array}
\usepackage{tabularx}

% 日本語フォント(デフォルト設定)
\renewcommand{\familydefault}{\sfdefault}

% ========================================
% Listings 環境設定(コード表示)
% ========================================
\lstset{
  language=Python,
  basicstyle=\ttfamily\small,
  keywordstyle=\bfseries\color{blue},
  commentstyle=\itshape\color{gray},
  stringstyle=\color{red},
  breaklines=true,
  breakatwhitespace=false,
  showstringspaces=false,
  numbers=left,
  numberstyle=\tiny\color{gray},
  firstnumber=1,
  frame=single,
  framerule=0pt,
  rulecolor=\color{lightgray},
  backgroundcolor=\color{white},
  xleftmargin=1cm,
  xrightmargin=0.5cm,
  captionpos=b,
  belowcaptionskip=0.5cm,
  lineskip=-1pt,
}

% YAML 用 listings スタイル
\lstdefinelanguage{yaml}{
  morecomment=[l]{\#},
  morestring=[b]{"},
  morestring=[b]{'},
  basicstyle=\ttfamily\small,
  keywordstyle=\bfseries\color{blue},
}

% JSON 用 listings スタイル
\lstdefinelanguage{json}{
  morestring=[b]{"},
  morecomment=[l]{//},
  basicstyle=\ttfamily\small,
}

% ========================================
% ヘッダー・フッター設定
% ========================================
\pagestyle{fancy}
\fancyhf{}
\fancyhead[L]{Copilot 完全自動化ガイド}
\fancyhead[R]{最終版}
\fancyfoot[C]{\thepage\ / \pageref{LastPage}}
\renewcommand{\headrulewidth}{0.4pt}
\renewcommand{\footrulewidth}{0.4pt}

% ========================================
% マクロ定義
% ========================================
\newcommand{\cmd}[1]{\texttt{#1}}
\newcommand{\file}[1]{\texttt{#1}}
\newcommand{\note}[1]{\textbf{注:} #1}
\newcommand{\warning}[1]{\textcolor{red}{\textbf{⚠ 警告:}} #1}
\newcommand{\tip}[1]{\textcolor{green}{\textbf{💡 ヒント:}} #1}

% ========================================
% ドキュメント開始
% ========================================
\title{\Huge\bfseries Copilot 完全自動化ガイド \\ \Large 最終版}
\author{Copilot Automation Guide Contributors}
\date{\today}

\begin{document}

% ========================================
% タイトルページ
% ========================================
\maketitle

\begin{abstract}
このドキュメントは、GitHub Copilot、GitHub Actions、Docker、Power Automate を組み合わせた
\textbf{完全自動化ガイド}です。本番環境対応のセキュリティベストプラクティスを実装した
CI/CD パイプライン、LaTeX PDF 自動生成、OCR 処理の自動化を実現します。

\vspace{1em}

\textbf{対象読者}: DevOps エンジニア、自動化担当者、Copilot を活用したい開発者

\vspace{1em}

\textbf{主な内容}:
\begin{itemize}
  \item スターターキット構成と使用方法
  \item ZIP スクリプトによるパッケージング
  \item GitHub Actions YAML ワークフロー
  \item Dockerfile による OCR 環境構築
  \item Power Automate 統合
  \item セキュリティベストプラクティス
  \item 実装手順書
  \item ワークフロー図と README 内容
\end{itemize}
\end{abstract}

% ========================================
% 目次
% ========================================
\newpage
\tableofcontents
\newpage

% ========================================
% セクション読み込み
% ========================================

% 01-overview.tex
% 概要セクション

\chapter{概要}

\section{このプロジェクトについて}

Copilot 完全自動化ガイドは、最新の DevOps ツールとクラウドサービスを組み合わせた
\textbf{エンドツーエンド自動化ソリューション}です。

\subsection{主な目標}

\begin{enumerate}
  \item LaTeX ドキュメントから PDF を自動生成
  \item スターターキットを ZIP にパッケージング
  \item GitHub Actions による CI/CD パイプライン構築
  \item Docker コンテナで OCR 処理を実行
  \item Power Automate で業務フロー自動化
  \item セキュリティベストプラクティス実装
\end{enumerate}

\section{アーキテクチャ概要}

\subsection{システム構成}

\begin{figure}[h]
\centering
\begin{verbatim}
┌─────────────────────────────────────────────┐
│   GitHub Repository (Trigger)               │
│   - Push, Pull Request, Release Tag         │
└────────────┬────────────────────────────────┘
             │
             ▼
┌─────────────────────────────────────────────┐
│   GitHub Actions Workflow                   │
│   - LaTeX Build (pdflatex x2)              │
│   - Docker Build & Push (GHCR)             │
│   - Starter Kit ZIP Generation             │
│   - Release Upload                         │
│   - Power Automate Trigger                 │
└────┬────────────────────────────┬───────────┘
     │                            │
     ▼                            ▼
┌──────────────────┐    ┌──────────────────┐
│  GitHub Release  │    │  GHCR Registry   │
│  (PDF + ZIP)     │    │  (Docker Image)  │
└──────────────────┘    └──────────────────┘
     │
     ▼
┌──────────────────────────────────────────────┐
│   Power Automate Flow                        │
│   - Download Release ZIP                     │
│   - OCR Processing (Docker)                  │
│   - CSV Generation                           │
│   - OneDrive Upload                          │
└──────────────────────────────────────────────┘
\end{verbatim}
\caption{システム全体構成図}
\end{figure}

\section{必要な事前知識}

このガイドを最大限に活用するには、以下の知識があると便利です:

\begin{itemize}
  \item \textbf{Git/GitHub}: リポジトリ管理、ワークフロー理解
  \item \textbf{YAML}: GitHub Actions の構文理解
  \item \textbf{Docker}: コンテナ基本、Dockerfile 記述
  \item \textbf{Python}: OCR スクリプト理解
  \item \textbf{LaTeX}: ドキュメント作成(初心者OK)
  \item \textbf{Power Automate}: クラウドフロー設計
  \item \textbf{bash/PowerShell}: シェルスクリプト実行
\end{itemize}

\section{対応環境}

\subsection{開発環境}

\begin{table}[h]
\centering
\begin{tabularx}{\textwidth}{|l|X|}
\hline
\textbf{環境} & \textbf{推奨バージョン} \\
\hline
Git & 2.20+ \\
Docker & 20.10+ \\
Python & 3.9+ \\
Node.js & 14+ (オプション) \\
LaTeX (TeX Live) & 2020 以上 \\
bash & 4.0+ \\
PowerShell & 5.1+ (Windows) \\
\hline
\end{tabularx}
\caption{推奨環境バージョン}
\end{table}

\subsection{クラウドサービス}

\begin{itemize}
  \item \textbf{GitHub}: Public or Private Repository
  \item \textbf{GitHub Packages (GHCR)}: Docker Image Registry
  \item \textbf{Microsoft OneDrive}: ファイルストレージ
  \item \textbf{Power Automate}: ビジネスプロセスオートメーション
\end{itemize}

\section{プロジェクト構成}

完全な自動化には以下のコンポーネントが含まれます:

\begin{lstlisting}[language=bash,caption=プロジェクトフォルダ構成]
copilot-automation-guide/
├── README.md                    # メインドキュメント
├── LICENSE                      # MIT ライセンス
├── .gitignore                   # Git 除外設定
├── requirements.txt             # Python 依存
├── Dockerfile                   # Docker イメージ
│
├── .github/workflows/
│   └── build.yml                # CI/CD パイプライン
│
├── latex/
│   ├── guide.tex                # メイン TeX ファイル
│   └── sections/                # セクション分割
│       ├── 01-overview.tex
│       ├── 02-starter-kit.tex
│       └── ...
│
├── starter-kit/                 # ZIP 化対象
│   ├── sample-config.yaml
│   ├── templates/
│   └── docs/
│
├── power-automate/
│   ├── flow.json                # フロー定義
│   └── README.md
│
└── scripts/
    └── generate-starter-kit.sh  # ZIP 生成
\end{lstlisting}

\section{本ガイドの使用方法}

\begin{enumerate}
  \item \textbf{セクション 1-3}: 基礎知識と構成理解
  \item \textbf{セクション 4-6}: 技術実装(YAML、Docker、Power Automate)
  \item \textbf{セクション 7}: セキュリティベストプラクティス
  \item \textbf{セクション 8-10}: 実装手順、ワークフロー、README 応用
\end{enumerate}

各セクションには実装例とコード例が含まれており、すぐに活用できます。

\tip{GitHub Copilot を使用して本ガイド内のコード例を拡張・カスタマイズできます。}

\newpage

% 02-starter-kit.tex
% スターターキット構成セクション

\chapter{スターターキット構成}

\section{スターターキットの概要}

スターターキット (\file{starter-kit/}) は、プロジェクト開始時に必要な全テンプレート、
設定ファイル、ドキュメントをまとめたパッケージです。

\subsection{目的}

\begin{itemize}
  \item 新しいプロジェクト立ち上げの高速化
  \item 標準的なファイル構造の提供
  \item ベストプラクティスの実装例
  \item セキュリティ設定の自動化
\end{itemize}

\section{ディレクトリ構造}

\begin{lstlisting}[language=bash,caption=スターターキットの構成]
starter-kit/
├── README.md                      # スターターキット ガイド
├── CONTRIBUTING.md                # 貢献ガイドライン
│
├── config/
│   ├── github-actions.yaml        # GitHub Actions テンプレート
│   ├── docker-compose.yml         # Docker Compose 設定
│   ├── .env.example               # 環境変数テンプレート
│   └── security-policy.md         # セキュリティポリシー
│
├── templates/
│   ├── issue-template.md          # Issue テンプレート
│   ├── pr-template.md             # Pull Request テンプレート
│   ├── dockerfile.template        # Dockerfile テンプレート
│   ├── workflow.yaml.template     # ワークフロー YAML テンプレート
│   └── python-main.py.template    # Python スクリプト テンプレート
│
├── scripts/
│   ├── setup.sh                   # 初期セットアップ スクリプト
│   ├── validate.sh                # 設定検証スクリプト
│   ├── deploy.sh                  # デプロイスクリプト
│   └── README.md                  # スクリプト ドキュメント
│
├── docs/
│   ├── GETTING_STARTED.md         # 始め方ガイド
│   ├── ARCHITECTURE.md            # アーキテクチャ説明
│   ├── DEPLOYMENT.md              # デプロイ手順
│   ├── TROUBLESHOOTING.md         # トラブルシューティング
│   └── FAQ.md                     # よくある質問
│
└── examples/
    ├── python/
    │   ├── main.py
    │   └── requirements.txt
    ├── bash/
    │   ├── example.sh
    │   └── utilities.sh
    └── yaml/
        ├── workflow-example.yaml
        └── docker-compose-example.yml
\end{lstlisting}

\section{各ファイルの説明}

\subsection{設定ファイル (config/)}

\begin{description}
  \item[\file{github-actions.yaml}] GitHub Actions ワークフロー設定テンプレート
  \item[\file{docker-compose.yml}] ローカル開発環境用 Docker Compose
  \item[\file{.env.example}] 環境変数テンプレート(本番では \file{.env} に改名)
  \item[\file{security-policy.md}] 脆弱性報告ポリシー
\end{description}

\subsection{テンプレート (templates/)}

\begin{lstlisting}[caption=Issue テンプレート例]
<!-- issue-template.md -->
## 問題の説明
<!-- 何が起きたか簡潔に説明 -->

## 再現手順
1. ステップ 1
2. ステップ 2
3. ステップ 3

## 期待される動作
<!-- 何が起こるべきだったか -->

## 実際の動作
<!-- 実際に起きたこと -->

## スクリーンショット(オプション)
<!-- スクリーンショットを添付 -->

## 環境情報
- OS: 
- Docker Version: 
- Python Version: 
\end{lstlisting}

\subsection{スクリプト (scripts/)}

\begin{table}[h]
\centering
\begin{tabularx}{\textwidth}{|l|X|}
\hline
\textbf{スクリプト} & \textbf{機能} \\
\hline
\file{setup.sh} & 環境の初期セットアップ(Python venv、依存関係インストール等) \\
\file{validate.sh} & 設定ファイル、環境変数、依存関係の検証 \\
\file{deploy.sh} & Docker イメージビルドと GHCR へのプッシュ \\
\hline
\end{tabularx}
\caption{スターターキット スクリプト}
\end{table}

\section{使用開始手順}

\subsection{ステップ 1: ZIP から展開}

\begin{lstlisting}[language=bash,caption=スターターキット展開]
unzip starter-kit.zip
cd starter-kit
\end{lstlisting}

\subsection{ステップ 2: 初期セットアップ実行}

\begin{lstlisting}[language=bash,caption=セットアップスクリプト実行]
bash scripts/setup.sh
\end{lstlisting}

\subsection{ステップ 3: 環境変数設定}

\begin{lstlisting}[language=bash,caption=環境変数ファイル作成]
cp config/.env.example .env
# .env をテキストエディタで編集して、実際の値を設定
\end{lstlisting}

\subsection{ステップ 4: 設定検証}

\begin{lstlisting}[language=bash,caption=設定検証実行]
bash scripts/validate.sh
\end{lstlisting}

\section{カスタマイズガイド}

\subsection{独自の GitHub Actions ワークフロー作成}

\begin{lstlisting}[language=yaml,caption=カスタム ワークフロー例]
name: Custom Build Pipeline

on:
  push:
    branches: [ main ]

jobs:
  build:
    runs-on: ubuntu-latest
    steps:
      - uses: actions/checkout@v3
      - name: Set up Python
        uses: actions/setup-python@v4
        with:
          python-version: '3.11'
      - name: Run tests
        run: python -m pytest
\end{lstlisting}

\subsection{Docker イメージのカスタマイズ}

\begin{lstlisting}[language=dockerfile,caption=カスタム Dockerfile]
FROM python:3.11-slim

# 作業ディレクトリ設定
WORKDIR /app

# 依存関係インストール
COPY requirements.txt .
RUN pip install --no-cache-dir -r requirements.txt

# アプリケーション コピー
COPY . .

# デフォルト コマンド
CMD ["python", "main.py"]
\end{lstlisting}

\section{ベストプラクティス}

\begin{itemize}
  \item \textbf{バージョン管理}: スターターキットも Git で管理する
  \item \textbf{ドキュメント}: 変更時は対応するドキュメントも更新
  \item \textbf{テスト}: 新規スクリプト追加時は自動テストを含める
  \item \textbf{セキュリティ}: センシティブ情報は環境変数に移行
\end{itemize}

\tip{スターターキットは GitHub Templates リポジトリとしても利用できます。}

\newpage

% 03-zip-script.tex
% ZIP スクリプトセクション

\chapter{ZIP スクリプトと自動パッケージング}

\section{概要}

ZIP スクリプトは \file{starter-kit/} フォルダを圧縮し、GitHub Release にアップロード可能な
パッケージを生成します。

\section{bash スクリプト実装}

\begin{lstlisting}[language=bash,caption=generate-starter-kit.sh - ZIP生成スクリプト]
#!/bin/bash
# generate-starter-kit.sh
# スターターキットを ZIP ファイルにパッケージング
# 作成日: 2025-12-10
# 対応: bash 4.0+, Windows (WSL/Git Bash)

set -euo pipefail

# ===================================
# 設定
# ===================================
SCRIPT_DIR="$(cd "$(dirname "${BASH_SOURCE[0]}")" && pwd)"
PROJECT_ROOT="$(dirname "$SCRIPT_DIR")"
STARTER_KIT_DIR="$PROJECT_ROOT/starter-kit"
OUTPUT_DIR="$PROJECT_ROOT/output"
TIMESTAMP=$(date +%Y%m%d_%H%M%S)
ZIP_NAME="starter-kit_${TIMESTAMP}.zip"
ZIP_PATH="$OUTPUT_DIR/$ZIP_NAME"

# ===================================
# 色付き出力
# ===================================
RED='\033[0;31m'
GREEN='\033[0;32m'
YELLOW='\033[1;33m'
BLUE='\033[0;34m'
NC='\033[0m' # No Color

# ===================================
# ログ関数
# ===================================
log_info() {
    echo -e "${BLUE}[INFO]${NC} $1"
}

log_success() {
    echo -e "${GREEN}[SUCCESS]${NC} $1"
}

log_error() {
    echo -e "${RED}[ERROR]${NC} $1" >&2
}

log_warning() {
    echo -e "${YELLOW}[WARNING]${NC} $1"
}

# ===================================
# 前提条件チェック
# ===================================
check_prerequisites() {
    log_info "前提条件をチェック中..."
    
    # zip コマンド確認
    if ! command -v zip &> /dev/null; then
        log_error "zip コマンドが見つかりません。インストールしてください。"
        exit 1
    fi
    
    # スターターキット ディレクトリ確認
    if [ ! -d "$STARTER_KIT_DIR" ]; then
        log_error "スターターキット ディレクトリ $STARTER_KIT_DIR が見つかりません。"
        exit 1
    fi
    
    log_success "前提条件チェック完了"
}

# ===================================
# 出力ディレクトリ作成
# ===================================
prepare_output_dir() {
    log_info "出力ディレクトリを準備中..."
    
    if [ ! -d "$OUTPUT_DIR" ]; then
        mkdir -p "$OUTPUT_DIR"
        log_info "出力ディレクトリを作成: $OUTPUT_DIR"
    fi
    
    log_success "出力ディレクトリ準備完了"
}

# ===================================
# ファイル検証
# ===================================
validate_files() {
    log_info "スターターキット内のファイルを検証中..."
    
    local required_files=(
        "README.md"
        "config"
        "templates"
        "scripts"
        "docs"
        "examples"
    )
    
    for file in "${required_files[@]}"; do
        if [ ! -e "$STARTER_KIT_DIR/$file" ]; then
            log_warning "推奨ファイル/ディレクトリが見つかりません: $file"
        else
            log_info "✓ $file"
        fi
    done
    
    log_success "ファイル検証完了"
}

# ===================================
# ZIP ファイル生成
# ===================================
create_zip() {
    log_info "ZIP ファイルを生成中..."
    log_info "対象ディレクトリ: $STARTER_KIT_DIR"
    log_info "出力先: $ZIP_PATH"
    
    # 既存の ZIP を削除
    if [ -f "$ZIP_PATH" ]; then
        log_warning "既存 ZIP を削除: $ZIP_PATH"
        rm -f "$ZIP_PATH"
    fi
    
    # ZIP 生成
    cd "$PROJECT_ROOT"
    zip -r -q "$ZIP_PATH" "starter-kit/" \
        -x "starter-kit/.DS_Store" \
           "starter-kit/*/.DS_Store" \
           "starter-kit/__pycache__/*" \
           "starter-kit/*.pyc"
    
    if [ -f "$ZIP_PATH" ]; then
        log_success "ZIP ファイル生成成功: $ZIP_PATH"
        
        # ファイルサイズ表示
        local size=$(du -h "$ZIP_PATH" | cut -f1)
        log_info "ファイルサイズ: $size"
    else
        log_error "ZIP ファイル生成に失敗しました。"
        exit 1
    fi
}

# ===================================
# ZIP 内容確認
# ===================================
verify_zip() {
    log_info "ZIP ファイルの内容を検証中..."
    
    local file_count=$(unzip -l "$ZIP_PATH" | tail -1 | awk '{print $2}')
    log_info "ZIP に含まれるファイル数: $file_count"
    
    # 重要ファイル確認
    if unzip -l "$ZIP_PATH" | grep -q "README.md"; then
        log_success "✓ README.md が含まれています"
    else
        log_warning "⚠ README.md が見つかりません"
    fi
    
    log_success "ZIP 検証完了"
}

# ===================================
# チェックサム生成
# ===================================
generate_checksum() {
    log_info "チェックサムを生成中..."
    
    local checksum_file="$OUTPUT_DIR/${ZIP_NAME%.zip}.sha256"
    
    if command -v sha256sum &> /dev/null; then
        sha256sum "$ZIP_PATH" > "$checksum_file"
        log_success "SHA256 チェックサム生成: $checksum_file"
    elif command -v shasum &> /dev/null; then
        shasum -a 256 "$ZIP_PATH" > "$checksum_file"
        log_success "SHA256 チェックサム生成: $checksum_file"
    else
        log_warning "チェックサム生成ツールが見つかりません。スキップします。"
    fi
}

# ===================================
# メイン処理
# ===================================
main() {
    log_info "========================================"
    log_info "  Starter Kit ZIP 生成スクリプト"
    log_info "========================================"
    
    check_prerequisites
    prepare_output_dir
    validate_files
    create_zip
    verify_zip
    generate_checksum
    
    log_info "========================================"
    log_success "スターターキットの ZIP 生成が完了しました!"
    log_info "出力: $ZIP_PATH"
    log_info "========================================"
}

# エラートラップ
trap 'log_error "スクリプト実行中にエラーが発生しました"; exit 1' ERR

# メイン実行
main "$@"
\end{lstlisting}

\section{スクリプト実行方法}

\subsection{Linux/macOS}

\begin{lstlisting}[language=bash,caption=Linux/macOS での実行]
chmod +x scripts/generate-starter-kit.sh
bash scripts/generate-starter-kit.sh
\end{lstlisting}

\subsection{Windows (PowerShell)}

\begin{lstlisting}[language=bash,caption=Windows PowerShell での実行]
# WSL 環境がある場合
wsl bash scripts/generate-starter-kit.sh

# または Git Bash で実行
bash scripts/generate-starter-kit.sh
\end{lstlisting}

\section{出力ファイル}

スクリプト実行後、以下のファイルが \file{output/} に生成されます:

\begin{table}[h]
\centering
\begin{tabularx}{\textwidth}{|l|X|}
\hline
\textbf{ファイル} & \textbf{説明} \\
\hline
\file{starter-kit\_YYYYMMDD\_HHMMSS.zip} & スターターキット ZIP \\
\file{starter-kit\_YYYYMMDD\_HHMMSS.sha256} & SHA256 チェックサム \\
\hline
\end{tabularx}
\caption{生成されるファイル}
\end{table}

\section{ベストプラクティス}

\begin{itemize}
  \item \textbf{定期実行}: CI/CD パイプラインで自動実行
  \item \textbf{チェックサム}: 整合性確認に SHA256 チェックサムを使用
  \item \textbf{ログ確認}: エラー時は詳細ログで原因を特定
  \item \textbf{バージョン管理}: ZIP ファイルは Git LFS で管理
\end{itemize}

\note{Windows 環境では、ZIP 生成前に 7-Zip や WinRAR などのツールで
テストすることをお勧めします。}

\newpage

% 04-github-actions.tex
% GitHub Actions セクション

\chapter{GitHub Actions YAML ワークフロー}

\section{概要}

GitHub Actions ワークフローは、リポジトリへのプッシュやリリース作成時に
自動的に実行される CI/CD パイプラインです。

\section{完全な build.yml 実装}

\begin{lstlisting}[language=yaml,caption=.github/workflows/build.yml - 完全版]
name: Build and Deploy Pipeline

# トリガー条件
on:
  push:
    branches:
      - main
      - develop
    tags:
      - 'v*'
  pull_request:
    branches:
      - main
  workflow_dispatch:  # 手動トリガー

# 環境変数(グローバル)
env:
  REGISTRY: ghcr.io
  IMAGE_NAME: ${{ github.repository }}
  LATEX_VERSION: texlive-2024

# 並列ジョブ実行
jobs:
  # ========================================
  # Job 1: LaTeX PDF ビルド
  # ========================================
  build-latex:
    name: Build LaTeX PDF
    runs-on: ubuntu-latest
    outputs:
      pdf_artifact: ${{ steps.build.outputs.pdf_name }}
    
    steps:
      # チェックアウト
      - name: Checkout repository
        uses: actions/checkout@v4
      
      # LaTeX 環境キャッシュ
      - name: Cache TeX Live
        uses: actions/cache@v3
        with:
          path: ~/.texlive2024/texmf-var
          key: ${{ runner.os }}-texlive-${{ env.LATEX_VERSION }}-${{ hashFiles('**/preamble.tex') }}
          restore-keys: |
            ${{ runner.os }}-texlive-${{ env.LATEX_VERSION }}-
      
      # LaTeX インストール
      - name: Install LaTeX dependencies
        run: |
          sudo apt-get update
          sudo apt-get install -y texlive-full texlive-xetex texlive-lang-japanese
          sudo apt-get install -y fonts-noto-cjk
      
      # PDF ビルド(2 回実行:目次生成用)
      - name: Build LaTeX PDF
        id: build
        run: |
          cd latex
          pdflatex -interaction=nonstopmode -halt-on-error guide.tex
          pdflatex -interaction=nonstopmode -halt-on-error guide.tex
          echo "pdf_name=guide.pdf" >> $GITHUB_OUTPUT
          ls -lah *.pdf
      
      # ビルド成功確認
      - name: Verify PDF generation
        run: |
          if [ -f "latex/guide.pdf" ]; then
            echo "✓ PDF ビルド成功"
            file latex/guide.pdf
          else
            echo "✗ PDF ビルド失敗"
            exit 1
          fi
      
      # PDF をアーティファクトとして保存
      - name: Upload PDF artifact
        uses: actions/upload-artifact@v3
        with:
          name: latex-pdf
          path: latex/guide.pdf
          retention-days: 30

  # ========================================
  # Job 2: Docker イメージビルド
  # ========================================
  build-docker:
    name: Build Docker Image
    runs-on: ubuntu-latest
    permissions:
      contents: read
      packages: write
    
    steps:
      - name: Checkout repository
        uses: actions/checkout@v4
      
      # Docker メタデータ生成
      - name: Extract metadata
        id: meta
        uses: docker/metadata-action@v5
        with:
          images: ${{ env.REGISTRY }}/${{ env.IMAGE_NAME }}
          tags: |
            type=semver,pattern={{version}}
            type=semver,pattern={{major}}.{{minor}}
            type=ref,event=branch
            type=sha,prefix={{branch}}-
      
      # Docker buildx セットアップ
      - name: Set up Docker Buildx
        uses: docker/setup-buildx-action@v2
      
      # GHCR ログイン
      - name: Log in to Container Registry
        uses: docker/login-action@v2
        with:
          registry: ${{ env.REGISTRY }}
          username: ${{ github.actor }}
          password: ${{ secrets.GITHUB_TOKEN }}
      
      # Docker イメージビルド & プッシュ
      - name: Build and push Docker image
        uses: docker/build-push-action@v4
        with:
          context: .
          push: true
          tags: ${{ steps.meta.outputs.tags }}
          labels: ${{ steps.meta.outputs.labels }}
          cache-from: type=gha
          cache-to: type=gha,mode=max
          build-args: |
            BUILD_DATE=${{ github.event.head_commit.timestamp }}
            VCS_REF=${{ github.sha }}

  # ========================================
  # Job 3: スターターキット生成
  # ========================================
  create-starter-kit:
    name: Create Starter Kit ZIP
    runs-on: ubuntu-latest
    outputs:
      zip_path: ${{ steps.create.outputs.zip_path }}
    
    steps:
      - name: Checkout repository
        uses: actions/checkout@v4
      
      - name: Create ZIP file
        id: create
        run: |
          OUTPUT_DIR="output"
          mkdir -p "$OUTPUT_DIR"
          
          ZIP_NAME="starter-kit_${{ github.run_id }}.zip"
          ZIP_PATH="$OUTPUT_DIR/$ZIP_NAME"
          
          cd ..
          zip -r -q "${{ github.workspace }}/$ZIP_PATH" \
            "copilot-automation-guide/starter-kit/" \
            -x "*/.*" "*/__pycache__/*" "*.pyc"
          
          echo "zip_path=$ZIP_PATH" >> $GITHUB_OUTPUT
          ls -lah "$ZIP_PATH"
      
      - name: Upload ZIP artifact
        uses: actions/upload-artifact@v3
        with:
          name: starter-kit
          path: ${{ steps.create.outputs.zip_path }}

  # ========================================
  # Job 4: GitHub Release にアップロード
  # ========================================
  upload-release:
    name: Upload to GitHub Release
    runs-on: ubuntu-latest
    needs: [build-latex, create-starter-kit]
    if: startsWith(github.ref, 'refs/tags/')
    
    steps:
      - name: Checkout repository
        uses: actions/checkout@v4
      
      - name: Download PDF artifact
        uses: actions/download-artifact@v3
        with:
          name: latex-pdf
          path: release-files
      
      - name: Download Starter Kit artifact
        uses: actions/download-artifact@v3
        with:
          name: starter-kit
          path: release-files
      
      - name: Create Release
        uses: softprops/action-gh-release@v1
        with:
          files: release-files/*
          draft: false
          prerelease: false
          generate_release_notes: true
        env:
          GITHUB_TOKEN: ${{ secrets.GITHUB_TOKEN }}

  # ========================================
  # Job 5: Power Automate トリガー
  # ========================================
  trigger-power-automate:
    name: Trigger Power Automate Flow
    runs-on: ubuntu-latest
    needs: upload-release
    if: startsWith(github.ref, 'refs/tags/')
    
    steps:
      - name: Get Release Info
        id: release
        run: |
          RELEASE_TAG=${{ github.ref }}
          RELEASE_TAG=${RELEASE_TAG#refs/tags/}
          echo "tag=$RELEASE_TAG" >> $GITHUB_OUTPUT
      
      - name: Trigger Power Automate Webhook
        run: |
          curl -X POST \
            "${{ secrets.POWER_AUTOMATE_WEBHOOK }}" \
            -H "Content-Type: application/json" \
            -d '{
              "tag": "${{ steps.release.outputs.tag }}",
              "repository": "${{ github.repository }}",
              "releaseUrl": "${{ github.server_url }}/${{ github.repository }}/releases/tag/${{ steps.release.outputs.tag }}",
              "triggeredBy": "${{ github.actor }}"
            }'
        env:
          GITHUB_TOKEN: ${{ secrets.GITHUB_TOKEN }}

  # ========================================
  # Job 6: テストとバリデーション
  # ========================================
  validate:
    name: Validate Build Artifacts
    runs-on: ubuntu-latest
    needs: [build-latex, build-docker, create-starter-kit]
    
    steps:
      - name: Download all artifacts
        uses: actions/download-artifact@v3
      
      - name: List artifacts
        run: |
          echo "=== Build Artifacts ===" 
          find . -type f -name "*.pdf" -o -name "*.zip" | sort
      
      - name: Verify file integrity
        run: |
          if [ ! -f "latex-pdf/guide.pdf" ]; then
            echo "✗ PDF が見つかりません"
            exit 1
          fi
          if [ ! -f "starter-kit/"*.zip ]; then
            echo "✗ Starter Kit ZIP が見つかりません"
            exit 1
          fi
          echo "✓ すべてのアーティファクト確認完了"

# ========================================
# ワークフロー終了後の処理
# ========================================
on-workflow-end:
  name: Cleanup and Notify
  runs-on: ubuntu-latest
  if: always()
  needs: [build-latex, build-docker, create-starter-kit, validate]
  
  steps:
    - name: Send Slack Notification
      if: failure()
      uses: slackapi/slack-github-action@v1
      with:
        webhook-url: ${{ secrets.SLACK_WEBHOOK }}
        payload: |
          {
            "text": "Build Pipeline Failed",
            "details": "${{ github.server_url }}/${{ github.repository }}/actions/runs/${{ github.run_id }}"
          }
\end{lstlisting}

\section{ワークフロー実行フロー}

\begin{figure}[h]
\centering
\begin{verbatim}
Trigger Event
    │
    ├─→ build-latex (並列)
    │   ├─ Cache TeX Live
    │   ├─ Install Dependencies
    │   ├─ pdflatex (2x)
    │   └─ Upload Artifact
    │
    ├─→ build-docker (並列)
    │   ├─ Extract Metadata
    │   ├─ Setup Buildx
    │   ├─ Login to Registry
    │   └─ Build & Push Image
    │
    └─→ create-starter-kit (並列)
        ├─ Create ZIP
        └─ Upload Artifact
         
    All Jobs Complete
         │
         ├─→ validate (依存)
         │   ├─ Download Artifacts
         │   └─ Verify Integrity
         │
         └─→ upload-release (条件: Tag Push)
             ├─ Create GitHub Release
             └─ Upload Files
                  │
                  └─→ trigger-power-automate
                      └─ Call Webhook
\end{verbatim}
\caption{ワークフロー実行シーケンス}
\end{figure}

\section{シークレット設定}

GitHub Settings から以下のシークレットを設定:

\begin{table}[h]
\centering
\begin{tabularx}{\textwidth}{|l|X|l|}
\hline
\textbf{名前} & \textbf{値} & \textbf{優先度} \\
\hline
\cmd{GITHUB\_TOKEN} & 自動設定 & ⭐⭐⭐ \\
\cmd{POWER\_AUTOMATE\_WEBHOOK} & Webhook URL & ⭐⭐⭐ \\
\cmd{SLACK\_WEBHOOK} & Slack Webhook (オプション) & ⭐⭐ \\
\hline
\end{tabularx}
\caption{必要なシークレット}
\end{table}

\section{トラブルシューティング}

\subsection{LaTeX ビルド失敗}

\begin{lstlisting}[language=bash,caption=デバッグ対策]
# ローカルで pdflatex テスト
cd latex
pdflatex -interaction=nonstopmode -halt-on-error guide.tex 2>&1 | tail -20
\end{lstlisting}

\subsection{Docker ビルド失敗}

原因の可能性:
\begin{itemize}
  \item キャッシュレイヤー問題 → \cmd{cache-to: type=gha,mode=max} を確認
  \item レジストリログイン → \cmd{GITHUB\_TOKEN} 権限を確認
  \item ディスク容量 → ランナーの空き容量を確認
\end{itemize}

\tip{GitHub Actions ランナーは 14GB メモリ、160GB SSD を備えています。}

\newpage

% 05-dockerfile.tex
% Dockerfile セクション

\chapter{Dockerfile と OCR 環境}

\section{概要}

このセクションでは、Python + Tesseract OCR を備えた本番環境対応の Dockerfile を実装します。

% ================================================================================
% コマンド定義
% ================================================================================
\newcommand{\cmd}[1]{\texttt{#1}}  % コマンド表示用
\newcommand{\tip}[1]{\textbf{ヒント:} #1}  % ヒント表示用

\section{Dockerfile 実装}

\begin{lstlisting}[language=dockerfile,caption=本番対応 Dockerfile]
# Dockerfile
# Copilot 完全自動化ガイド - OCR 環境
# マルチステージビルド対応
# 作成日: 2025-12-10

# ========================================
# ステージ 1: ビルドステージ
# ========================================
FROM python:3.11-slim as builder

LABEL maintainer="Copilot Automation <info@example.com>"
LABEL description="OCR Processing Environment for Copilot Automation"
LABEL version="1.0.0"

# 作業ディレクトリ設定
WORKDIR /build

# ビルド依存関係インストール
RUN apt-get update && apt-get install -y --no-install-recommends \
    build-essential \
    gcc \
    g++ \
    git \
    && rm -rf /var/lib/apt/lists/*

# Python 依存関係ダウンロード
COPY requirements.txt .
RUN pip install --no-cache-dir --upgrade pip setuptools wheel && \
    pip download --no-cache-dir \
    --dest /pip-packages \
    -r requirements.txt

# ========================================
# ステージ 2: ランタイムステージ
# ========================================
FROM python:3.11-slim

# メタデータ
ARG BUILD_DATE
ARG VCS_REF
LABEL org.label-schema.build-date=$BUILD_DATE
LABEL org.label-schema.vcs-ref=$VCS_REF

# 非ルートユーザー作成
RUN groupadd -r appuser && useradd -r -g appuser appuser

# 作業ディレクトリ
WORKDIR /app

# システムパッケージをインストール(OCR)
RUN apt-get update && apt-get install -y --no-install-recommends \
    tesseract-ocr \
    tesseract-ocr-jpn \
    tesseract-ocr-eng \
    libtesseract-dev \
    leptonica-progs \
    libmagic1 \
    fonts-noto-cjk \
    && rm -rf /var/lib/apt/lists/*

# キャッシュマウント経由で効率的にインストール
RUN --mount=type=cache,target=/root/.cache/pip \
    pip install --upgrade pip setuptools wheel && \
    pip install --no-cache-dir \
    pytesseract==0.3.10 \
    pillow==10.0.0 \
    opencv-python==4.8.0.76 \
    numpy==1.24.3 \
    pandas==2.0.3 \
    pydantic==2.0.0

# 依存ファイル コピー(本番)
RUN --mount=type=cache,target=/root/.cache/pip \
    pip install --no-cache-dir -r requirements.txt || true

# アプリケーション コピー
COPY --chown=appuser:appuser . .

# ディレクトリパーミッション設定
RUN mkdir -p /app/input /app/output && \
    chown -R appuser:appuser /app && \
    chmod 755 /app/input /app/output

# 非ルートユーザーに切り替え
USER appuser

# ヘルスチェック
HEALTHCHECK --interval=30s --timeout=10s --start-period=5s --retries=3 \
    CMD python -c "import pytesseract; pytesseract.get_tesseract_version()" || exit 1

# デフォルトコマンド
ENTRYPOINT ["python"]
CMD ["main.py"]

# ========================================
# ステージ 3: テストステージ(オプション)
# ========================================
FROM python:3.11-slim as test

WORKDIR /app

# テストツール
RUN apt-get update && apt-get install -y --no-install-recommends \
    pytest \
    pytest-cov \
    && rm -rf /var/lib/apt/lists/*

COPY --from=builder /pip-packages /pip-packages
RUN pip install /pip-packages/* && \
    pip install pytest pytest-cov

COPY . .

# テスト実行
CMD ["pytest", "-v", "--cov=.", "tests/"]
\end{lstlisting}

\section{requirements.txt}

\begin{lstlisting}[caption=requirements.txt - Python 依存関係]
# requirements.txt
# Python OCR 環境 依存関係
# Python 3.9+

# === OCR & 画像処理 ===
pytesseract==0.3.10          # Tesseract Python インターフェース
pillow==10.0.0               # 画像処理ライブラリ
opencv-python==4.8.0.76      # コンピュータビジョン
numpy==1.24.3                # 数値計算

# === データ処理 ===
pandas==2.0.3                # データ分析
openpyxl==3.1.2             # Excel 処理
pydantic==2.0.0              # データバリデーション

# === ファイルハンドリング ===
python-magic-bin==0.4.14    # ファイル型認識

# === ロギング & モニタリング ===
python-json-logger==2.0.7   # JSON ロギング
prometheus-client==0.17.1   # Prometheus メトリクス

# === Web & API ===
requests==2.31.0            # HTTP ライブラリ
httpx==0.24.1               # 非同期 HTTP

# === 設定管理 ===
python-dotenv==1.0.0        # .env ファイル読み込み

# === デバッグ & 開発 ===
pytest==7.4.0               # テストフレームワーク(開発用)
pytest-cov==4.1.0           # カバレッジ測定
black==23.7.0               # コードフォーマッター
flake8==6.0.0               # リンター

# === セキュリティ ===
cryptography==41.0.3        # 暗号化
\end{lstlisting}

\section{Docker 実行方法}

\subsection{ローカルビルド}

\begin{lstlisting}[language=bash,caption=Docker イメージビルド]
# イメージビルド
docker build -t copilot-automation:latest .

# タグ付け(GHCR 用)
docker tag copilot-automation:latest \
  ghcr.io/your-org/copilot-automation:latest
\end{lstlisting}

\subsection{コンテナ実行(OCR 処理例)}

\begin{lstlisting}[language=bash,caption=Docker コンテナ実行]
# 入出力ボリュームマウント
docker run --rm \
  --name copilot-ocr \
  -v $(pwd)/input:/app/input \
  -v $(pwd)/output:/app/output \
  copilot-automation:latest \
  main.py

# 環境変数指定
docker run --rm \
  -e LOG_LEVEL=DEBUG \
  -e OCR_LANGUAGE=jpn+eng \
  -v $(pwd)/input:/app/input \
  -v $(pwd)/output:/app/output \
  copilot-automation:latest
\end{lstlisting}

\subsection{Docker Compose(開発環境)}

\begin{lstlisting}[language=yaml,caption=docker-compose.yml]
version: '3.8'

services:
  ocr-processor:
    build:
      context: .
      dockerfile: Dockerfile
    container_name: copilot-ocr
    environment:
      - LOG_LEVEL=INFO
      - OCR_LANGUAGE=jpn+eng
      - TZ=Asia/Tokyo
    volumes:
      - ./input:/app/input
      - ./output:/app/output
      - ./logs:/app/logs
    networks:
      - copilot-net
    restart: on-failure

  # オプション: nginx リバースプロキシ
  nginx:
    image: nginx:alpine
    ports:
      - "80:80"
    volumes:
      - ./output:/usr/share/nginx/html:ro
    depends_on:
      - ocr-processor
    networks:
      - copilot-net

networks:
  copilot-net:
    driver: bridge
\end{lstlisting}

\section{main.py - OCR スクリプト例}

\begin{lstlisting}[language=python,caption=main.py - OCR メイン処理]
#!/usr/bin/env python3
# -*- coding: utf-8 -*-
"""
main.py - OCR 処理メインスクリプト
Tesseract OCR を使用してPDF/画像からテキストを抽出し CSV に出力
"""

import os
import sys
import logging
from pathlib import Path
import pytesseract
from PIL import Image
import pandas as pd

# ロギング設定
logging.basicConfig(
    level=os.getenv('LOG_LEVEL', 'INFO'),
    format='%(asctime)s - %(name)s - %(levelname)s - %(message)s'
)
logger = logging.getLogger(__name__)

# 入出力パス
INPUT_DIR = Path('/app/input')
OUTPUT_DIR = Path('/app/output')

def extract_text_from_image(image_path: Path) -> str:
    """
    画像ファイルから OCR でテキスト抽出
    
    Args:
        image_path: 画像ファイルパス
    
    Returns:
        抽出されたテキスト
    """
    try:
        image = Image.open(image_path)
        # 日本語 + 英語で OCR
        text = pytesseract.image_to_string(
            image,
            lang='jpn+eng'
        )
        logger.info(f"✓ {image_path.name} から OCR 抽出完了")
        return text
    except Exception as e:
        logger.error(f"✗ OCR 抽出失敗 {image_path}: {e}")
        return ""

def process_input_directory():
    """入力ディレクトリ内の全ファイルを処理"""
    results = []
    
    # サポート形式
    image_extensions = {'.jpg', '.jpeg', '.png', '.tif', '.tiff'}
    
    for file_path in INPUT_DIR.glob('*'):
        if file_path.suffix.lower() not in image_extensions:
            logger.warning(f"スキップ: {file_path.name} (非対応形式)")
            continue
        
        logger.info(f"処理中: {file_path.name}")
        text = extract_text_from_image(file_path)
        
        results.append({
            'filename': file_path.name,
            'extracted_text': text,
            'text_length': len(text)
        })
    
    return results

def save_results_to_csv(results: list):
    """
    結果を CSV に保存
    
    Args:
        results: OCR 結果リスト
    """
    if not results:
        logger.warning("処理対象ファイルがありません")
        return
    
    df = pd.DataFrame(results)
    output_file = OUTPUT_DIR / 'ocr_results.csv'
    
    try:
        df.to_csv(output_file, index=False, encoding='utf-8-sig')
        logger.info(f"✓ CSV 保存成功: {output_file}")
        logger.info(f"  処理ファイル数: {len(results)}")
    except Exception as e:
        logger.error(f"✗ CSV 保存失敗: {e}")

def main():
    """メイン処理"""
    logger.info("=" * 50)
    logger.info("OCR 処理スクリプト開始")
    logger.info("=" * 50)
    
    # ディレクトリ確認
    if not INPUT_DIR.exists():
        logger.error(f"入力ディレクトリが見つかりません: {INPUT_DIR}")
        return 1
    
    OUTPUT_DIR.mkdir(exist_ok=True)
    
    # 処理実行
    results = process_input_directory()
    save_results_to_csv(results)
    
    logger.info("=" * 50)
    logger.info("OCR 処理完了")
    logger.info("=" * 50)
    return 0

if __name__ == '__main__':
    sys.exit(main())
\end{lstlisting}

\section{セキュリティベストプラクティス}

\begin{itemize}
  \item \textbf{マルチステージビルド}: 本番イメージサイズ最小化
  \item \textbf{非ルートユーザー}: \cmd{USER appuser} で権限制限
  \item \textbf{ヘルスチェック}: コンテナ健全性自動監視
  \item \textbf{キャッシュマウント}: ビルド高速化
  \item \textbf{.dockerignore}: センシティブファイル除外
\end{itemize}

\tip{イメージサイズは約 1.2GB(Tesseract + 日本語フォント込み)です。
本番環境では Amazon ECR や Docker Hub キャッシュを活用してください。}

\newpage

% 06-power-automate.tex
% Power Automate セクション

\chapter{Power Automate フロー統合}

\section{概要}

Power Automate は、GitHub Release からのファイルダウンロード、OCR 処理、
結果の OneDrive 保存を自動化するクラウドベースのワークフローエンジンです。

\section{フロー定義 JSON}

\begin{lstlisting}[language=json,caption=power-automate/flow.json]
{
  "displayName": "Copilot Automation - Release OCR Pipeline",
  "description": "GitHub Release ZIP をダウンロード、OCR 処理、
    結果を OneDrive に保存",
  "triggers": [
    {
      "name": "manual",
      "type": "Request",
      "kind": "Http",
      "properties": {
        "schema": {
          "$schema": "http://json-schema.org/draft-04/schema#",
          "type": "object",
          "properties": {
            "tag": {
              "type": "string",
              "description": "GitHub Release タグ"
            },
            "repository": {
              "type": "string",
              "description": "リポジトリ名"
            },
            "releaseUrl": {
              "type": "string",
              "description": "Release URL"
            }
          },
          "required": ["tag", "repository", "releaseUrl"]
        }
      }
    }
  ],
  "actions": [
    {
      "name": "Initialize Variables",
      "type": "InitializeVariable",
      "properties": {
        "variables": [
          {
            "name": "WorkspaceId",
            "type": "String",
            "value": "@triggerBody()?['repository']"
          },
          {
            "name": "ZipUrl",
            "type": "String",
            "value": ""
          },
          {
            "name": "OcrResults",
            "type": "Array",
            "value": []
          }
        ]
      }
    },
    {
      "name": "Get Release Assets",
      "type": "Http",
      "properties": {
        "method": "GET",
        "uri": "@concat('https://api.github.com/repos/',
          triggerBody()?['repository'],
          '/releases/tags/',
          triggerBody()?['tag'])",
        "headers": {
          "Accept": "application/vnd.github+json",
          "X-GitHub-Api-Version": "2022-11-28",
          "Authorization": "@concat('Bearer ',
            triggerBody()?['github_token'])"
        }
      }
    },
    {
      "name": "Parse Release Response",
      "type": "ParseJson",
      "properties": {
        "content": "@body('Get Release Assets')",
        "schema": {
          "type": "object",
          "properties": {
            "assets": {
              "type": "array",
              "items": {
                "type": "object",
                "properties": {
                  "name": {"type": "string"},
                  "browser_download_url": {"type": "string"},
                  "size": {"type": "integer"}
                }
              }
            }
          }
        }
      }
    },
    {
      "name": "Filter ZIP File",
      "type": "Filter",
      "properties": {
        "from": "@body('Parse Release Response')?['assets']",
        "where": "@endsWith(item()?['name'], '.zip')"
      }
    },
    {
      "name": "Download ZIP File",
      "type": "Http",
      "properties": {
        "method": "GET",
        "uri": "@body('Filter ZIP File')[0]?['browser_download_url']",
        "headers": {}
      }
    },
    {
      "name": "Create Temporary File",
      "type": "SharePointOnlineSendAnHTTPRequestToSharePoint",
      "properties": {
        "method": "POST",
        "uri": "@concat('https://www.sharepoint.com/sites/',
          variables('WorkspaceId'),
          '/_api/web/GetFileByServerRelativeUrl(',
          '''/sites/WorkspaceId/Shared Documents/temp-',
          utcNow('yyyyMMddHHmmss'),
          '.zip'',
          ')/StartUpload')",
        "body": "@body('Download ZIP File')"
      }
    },
    {
      "name": "Extract ZIP Contents",
      "type": "Http",
      "properties": {
        "method": "POST",
        "uri": "@concat(
          'https://your-docker-registry.azurewebsites.net/extract',
          '?zipUrl=',
          encodeUriComponent(body('Create Temporary File')?['uri']
          )
        )",
        "headers": {
          "Content-Type": "application/json",
          "Authorization": "@concat('Bearer ',
            parameters('DockerApiKey'))"
        },
        "body": {
          "action": "extract",
          "format": "zip"
        }
      }
    },
    {
      "name": "Process with OCR",
      "type": "Http",
      "properties": {
        "method": "POST",
        "uri": "@parameters('OcrServiceEndpoint')",
        "headers": {
          "Content-Type": "application/json",
          "Authorization": "@concat('Bearer ',
            parameters('OcrApiKey'))"
        },
        "body": {
          "operation": "ocr",
          "language": "jpn+eng",
          "sourceUri": "@body('Extract ZIP Contents')?['extractUri']"
        }
      }
    },
    {
      "name": "Convert to CSV",
      "type": "Html2text",
      "properties": {
        "html": "@body('Process with OCR')?['results']",
        "format": "csv",
        "encoding": "utf-8"
      }
    },
    {
      "name": "Save to OneDrive",
      "type": "SharePointOnlineSendAnHTTPRequestToSharePoint",
      "properties": {
        "method": "POST",
        "uri": "@concat(
          'https://graph.microsoft.com/v1.0/me/drive/root:',
          '/OCR_Results_',
          triggerBody()?['tag'],
          '.csv:/content'
        )",
        "headers": {
          "Content-Type": "text/csv",
          "Authorization": "@concat('Bearer ',
            parameters('MsGraphToken'))"
        },
        "body": "@body('Convert to CSV')"
      }
    },
    {
      "name": "Send Completion Notification",
      "type": "SendEmail",
      "properties": {
        "to": "@parameters('NotificationEmail')",
        "subject": "@concat(
          'OCR Processing Complete - ',
          triggerBody()?['tag']
        )",
        "body": "@concat(
          '<p>Processing completed successfully.</p>',
          '<p>Release: ',
          triggerBody()?['tag'],
          '</p>',
          '<p>Results saved to OneDrive.</p>'
        )"
      }
    }
  ],
  "outputs": [
    {
      "name": "ProcessingStatus",
      "type": "String",
      "value": "Completed"
    },
    {
      "name": "ResultFile",
      "type": "String",
      "value": "@outputs('Save to OneDrive')"
    }
  ]
}
\end{lstlisting}

\section{フロー構成図}

\begin{figure}[h]
\centering
\begin{verbatim}
┌──────────────────────────────────────────┐
│   Webhook トリガー                        │
│  (GitHub Actions から呼び出し)            │
└────────────────┬─────────────────────────┘
                 │
                 ▼
┌──────────────────────────────────────────┐
│   変数初期化                              │
│   - WorkspaceId                         │
│   - ZipUrl, OcrResults                 │
└────────────────┬─────────────────────────┘
                 │
                 ▼
┌──────────────────────────────────────────┐
│   GitHub API: Release 取得               │
│   GET /repos/{owner}/{repo}/releases   │
└────────────────┬─────────────────────────┘
                 │
                 ▼
┌──────────────────────────────────────────┐
│   ZIP ファイルをフィルタ                 │
│   (assets から .zip を検出)              │
└────────────────┬─────────────────────────┘
                 │
                 ▼
┌──────────────────────────────────────────┐
│   ZIP ダウンロード                        │
│   (ブラウザダウンロード URL)             │
└────────────────┬─────────────────────────┘
                 │
                 ▼
┌──────────────────────────────────────────┐
│   一時ファイル作成 (OneDrive)             │
└────────────────┬─────────────────────────┘
                 │
                 ▼
┌──────────────────────────────────────────┐
│   Docker サービス: ZIP 展開               │
│   (Docker HTTP エンドポイント)           │
└────────────────┬─────────────────────────┘
                 │
                 ▼
┌──────────────────────────────────────────┐
│   OCR 処理                                │
│   (Docker コンテナで実行)                │
│   言語: 日本語 + 英語                     │
└────────────────┬─────────────────────────┘
                 │
                 ▼
┌──────────────────────────────────────────┐
│   CSV 変換                                │
│   (HTML テーブル → CSV)                  │
└────────────────┬─────────────────────────┘
                 │
                 ▼
┌──────────────────────────────────────────┐
│   OneDrive 保存                           │
│   /OCR_Results_{tag}.csv                │
└────────────────┬─────────────────────────┘
                 │
                 ▼
┌──────────────────────────────────────────┐
│   完了通知メール送信                     │
└──────────────────────────────────────────┘
\end{verbatim}
\caption{Power Automate フロー実行フロー}
\end{figure}

\section{セットアップ手順}

\subsection{ステップ 1: Power Automate にインポート}

\begin{enumerate}
  \item Power Automate にログイン: \url{https://powerautomate.microsoft.com}
  \item \cmd{My flows} → \cmd{Create} → \cmd{Instant cloud flow} → \cmd{Automated}
  \item フロー名入力: \textit{Copilot Automation - Release OCR Pipeline}
  \item \cmd{Create}
  \item \cmd{+ New step} → \cmd{Parse JSON}
\end{enumerate}

\subsection{ステップ 2: コネクション設定}

\begin{table}[h]
\centering
\begin{tabularx}{\textwidth}{|l|X|}
\hline
\textbf{コネクション} & \textbf{設定} \\
\hline
GitHub & GitHub アカウント認証 \\
SharePoint & Office 365 アカウント \\
OneDrive & OneDrive アカウント \\
HTTP & Docker API キー \\
\hline
\end{tabularx}
\caption{必要なコネクション}
\end{table}

\subsection{ステップ 3: パラメータ設定}

\begin{lstlisting}[caption=フロー パラメータ]
DockerApiKey: <Docker Registry API キー>
OcrServiceEndpoint: https://your-docker.azurewebsites.net/api/ocr
OcrApiKey: <OCR サービス API キー>
MsGraphToken: <Microsoft Graph トークン>
NotificationEmail: admin@example.com
\end{lstlisting}

\section{エラーハンドリング}

\begin{lstlisting}[language=json,caption=エラー ハンドラーアクション]
{
  "name": "On Error Handler",
  "type": "Scope",
  "properties": {
    "runAfter": {
      "Process with OCR": ["Failed", "TimedOut"]
    },
    "actions": [
      {
        "name": "Log Error",
        "type": "Http",
        "properties": {
          "method": "POST",
          "uri": "@parameters('LoggingEndpoint')",
          "body": {
            "error": "@body('Process with OCR')",
            "timestamp": "@utcNow()"
          }
        }
      },
      {
        "name": "Send Error Notification",
        "type": "SendEmail",
        "properties": {
          "to": "@parameters('AdminEmail')",
          "subject": "OCR Processing Failed",
          "body": "@concat(
            '<p>Error: ',
            body('Process with OCR')?['error'],
            '</p>'
          )"
        }
      }
    ]
  }
}
\end{lstlisting}

\section{ベストプラクティス}

\begin{itemize}
  \item \textbf{タイムアウト管理}: 長時間処理には \cmd{timeout: 3600} を設定
  \item \textbf{リトライポリシー}: \cmd{retryPolicy} で自動リトライ有効化
  \item \textbf{ログ記録}: 全アクション出力を Azure Log Analytics に送信
  \item \textbf{通知}: 完了/エラーメール、Slack 通知の実装
  \item \textbf{セキュリティ}: シークレット値は \cmd{@parameters()} で参照
\end{itemize}

\tip{Power Automate Premium ライセンスで、
カスタムコネクタや高度なスケジューリングが利用可能です。}

\newpage

% 07-security.tex
% セキュリティセクション

\chapter{セキュリティベストプラクティス}

\section{概要}

本番環境における CI/CD パイプライン、コンテナ、クラウドサービスの
セキュリティは最優先です。本セクションでは実装済みのセキュリティ対策を詳述します。

\section{シークレット管理}

\subsection{GitHub Secrets}

\begin{lstlisting}[language=yaml,caption=シークレット参照(安全)]
# build.yml での使用方法

# ✓ 正しい方法:シークレットから参照
env:
  GITHUB_TOKEN: ${{ secrets.GITHUB_TOKEN }}
  PAT_TOKEN: ${{ secrets.PAT_TOKEN }}

jobs:
  deploy:
    steps:
      - name: Deploy
        env:
          # ✓ ステップレベルでシークレット参照
          DOCKER_PASSWORD: ${{ secrets.DOCKER_PASSWORD }}
        run: |
          # ✗ ログ出力禁止
          # echo $DOCKER_PASSWORD
          
          # ✓ マスク処理
          echo "::add-mask::$DOCKER_PASSWORD"
\end{lstlisting}

\subsection{環境変数とシークレット}

\begin{table}[h]
\centering
\begin{tabularx}{\textwidth}{|l|X|X|}
\hline
\textbf{タイプ} & \textbf{用途} & \textbf{管理方法} \\
\hline
シークレット & パスワード、トークン、API キー & GitHub Secrets \\
環境変数 & ビルド設定、ログレベル & .env ファイル \\
設定ファイル & アプリケーション設定 & config/ ディレクトリ \\
\hline
\end{tabularx}
\caption{情報管理の分類}
\end{table}

\section{Docker セキュリティ}

\subsection{マルチステージビルド}

\begin{lstlisting}[language=dockerfile,caption=セキュアな マルチステージビルド]
# ステージ 1: ビルド(開発ツール含む)
FROM python:3.11-slim as builder

WORKDIR /build

# ビルド依存(本番では不要)
RUN apt-get update && apt-get install -y gcc g++ make

COPY requirements.txt .
RUN pip install -r requirements.txt

# ステージ 2: ランタイム(最小限)
FROM python:3.11-slim

# 非ルートユーザー作成
RUN groupadd -r appuser && useradd -r -g appuser appuser

COPY --from=builder /usr/local/lib/python3.11/site-packages \
  /usr/local/lib/python3.11/site-packages

USER appuser  # 非ルートユーザーに切り替え

CMD ["python", "main.py"]
\end{lstlisting}

\subsection{イメージスキャンとセキュリティチェック}

\begin{lstlisting}[language=yaml,caption=trivy によるスキャン]
- name: Scan Docker Image with Trivy
  uses: aquasecurity/trivy-action@master
  with:
    image-ref: 'copilot-automation:latest'
    format: 'sarif'
    output: 'trivy-results.sarif'

- name: Upload Trivy results
  uses: github/codeql-action/upload-sarif@v2
  with:
    sarif_file: 'trivy-results.sarif'
\end{lstlisting}

\section{ネットワーク セキュリティ}

\subsection{HTTPS 強制}

\begin{lstlisting}[language=yaml,caption=HTTPS トラフィック強制]
jobs:
  deploy:
    steps:
      - name: Verify HTTPS
        run: |
          # URL スキーム検証
          if [[ ! "$WEBHOOK_URL" =~ ^https:// ]]; then
            echo "✗ HTTPS が必須です"
            exit 1
          fi
          echo "✓ HTTPS 確認完了"
        env:
          WEBHOOK_URL: ${{ secrets.POWER_AUTOMATE_WEBHOOK }}
\end{lstlisting}

\subsection{API キーローテーション}

\begin{itemize}
  \item \textbf{定期実行}: 30-90 日ごと
  \item \textbf{ローテーション方法}: 新キー追加 → 古キー削除 → 再デプロイ
  \item \textbf{自動化}: GitHub Actions スケジュール実行
  \item \textbf{監査}: 全ローテーション操作をログ記録
\end{itemize}

\section{コード セキュリティ}

\subsection{依存関係チェック}

\begin{lstlisting}[language=yaml,caption=Dependabot による脆弱性検出]
# .github/dependabot.yml
version: 2
updates:
  # Python 依存関係
  - package-ecosystem: "pip"
    directory: "/"
    schedule:
      interval: "daily"
    security-updates-only: true
    reviewers:
      - "security-team"

  # Docker ベースイメージ
  - package-ecosystem: "docker"
    directory: "/"
    schedule:
      interval: "weekly"

  # GitHub Actions
  - package-ecosystem: "github-actions"
    directory: "/"
    schedule:
      interval: "weekly"
\end{lstlisting}

\subsection{SAST (Static Application Security Testing)}

\begin{lstlisting}[language=yaml,caption=CodeQL による脆弱性解析]
- name: Initialize CodeQL
  uses: github/codeql-action/init@v2
  with:
    languages: 'python'

- name: Run CodeQL analysis
  uses: github/codeql-action/analyze@v2

- name: Upload CodeQL results
  uses: github/codeql-action/upload-sarif@v2
  with:
    sarif_file: 'codeql-results.sarif'
\end{lstlisting}

\section{ログと監査}

\subsection{セキュリティログ記録}

\begin{lstlisting}[language=python,caption=Python セキュリティログ]
import logging
import json
from datetime import datetime

# セキュアなロギング設定
logging.basicConfig(
    level=logging.INFO,
    format='%(asctime)s - %(name)s - %(levelname)s - %(message)s',
    handlers=[
        logging.FileHandler('security.log'),
        logging.StreamHandler()
    ]
)

logger = logging.getLogger(__name__)

def log_security_event(event_type, user, action, result):
    """セキュリティイベント ロギング"""
    log_entry = {
        'timestamp': datetime.utcnow().isoformat(),
        'event_type': event_type,
        'user': user,
        'action': action,
        'result': result
    }
    logger.info(json.dumps(log_entry))

# 使用例
log_security_event(
    event_type='deployment',
    user='github-actions',
    action='push_docker_image',
    result='success'
)
\end{lstlisting}

\subsection{監査トレイル}

\begin{itemize}
  \item \textbf{GitHub}: Settings → Audit log で全操作を監視
  \item \textbf{Docker}: GHCR → Package settings で push/pull ログを確認
  \item \textbf{OneDrive}: SharePoint → Site contents で ファイル変更追跡
  \item \textbf{Power Automate}: Run history で フロー実行履歴確認
\end{itemize}

\section{リスク評価と対応}

\subsection{セキュリティリスク行列}

\begin{table}[h]
\centering
\begin{tabularx}{\textwidth}{|l|l|l|l|}
\hline
\textbf{リスク} & \textbf{影響} & \textbf{可能性} & \textbf{対応} \\
\hline
シークレット漏洩 & 🔴 致命的 & 中 & GitHub Secrets \\
脆弱性 in deps & 🟠 高 & 高 & Dependabot \\
無許可アクセス & 🔴 致命的 & 低 & 2FA, OIDC \\
DoS 攻撃 & 🟡 中 & 低 & レート制限 \\
\hline
\end{tabularx}
\caption{セキュリティリスク評価}
\end{table}

\section{コンプライアンス}

\subsection{チェックリスト}

\begin{itemize}
  \item \checkbox{} GitHub Secret で全シークレット管理
  \item \checkbox{} Docker イメージ定期スキャン (Trivy)
  \item \checkbox{} Dependabot 有効化
  \item \checkbox{} CodeQL 分析実行
  \item \checkbox{} セキュリティログ記録
  \item \checkbox{} 監査トレイル確認
  \item \checkbox{} API キー 90 日ローテーション
  \item \checkbox{} HTTPS のみ使用
  \item \checkbox{} 非ルートユーザー実行
  \item \checkbox{} パーミッション最小化 (RBAC)
\end{itemize}

\note{本番環境では全項目をチェック状態にしてください。}

\newpage

% 08-manual.tex
% 実装手順セクション

\chapter{実装手順書}

\section{全体フロー}

本セクションでは、ゼロからプロジェクトを立ち上げるための
ステップバイステップガイドを提供します。

\subsection{前提条件確認}

\begin{lstlisting}[language=bash,caption=前提条件チェック]
#!/bin/bash

echo "=== 前提条件チェック ==="

# Git
git --version || { echo "✗ Git をインストールしてください"; exit 1; }

# Docker
docker --version || { echo "✗ Docker をインストールしてください"; exit 1; }

# Python
python3 --version || { echo "✗ Python 3.9+ をインストールしてください"; exit 1; }

# TeX Live (オプション)
pdflatex --version && echo "✓ LaTeX 環境あり" || echo "⚠ LaTeX なし (PDF 生成スキップ)"

echo "✓ 前提条件チェック完了"
\end{lstlisting}

\section{ステップ 1: GitHub リポジトリセットアップ}

\subsection{1-1: リポジトリ作成}

\begin{enumerate}
  \item GitHub にログイン: \url{https://github.com}
  \item \cmd{New} → リポジトリ作成
  \item リポジトリ名: \textit{copilot-automation-guide}
  \item 説明: \textit{Copilot Complete Automation Guide}
  \item Visibility: \textit{Public} (または Private)
  \item \cmd{Create repository}
\end{enumerate}

\subsection{1-2: リポジトリクローン}

\begin{lstlisting}[language=bash,caption=リポジトリクローン]
git clone https://github.com/YOUR-USERNAME/copilot-automation-guide.git
cd copilot-automation-guide

# ブランチ確認
git branch -a

# develop ブランチ作成(推奨)
git checkout -b develop
\end{lstlisting}

\subsection{1-3: 初期ファイル配置}

\begin{lstlisting}[language=bash,caption=ファイル配置]
# リポジトリルート
touch README.md LICENSE .gitignore

# ディレクトリ作成
mkdir -p .github/workflows
mkdir -p latex/sections
mkdir -p starter-kit/{config,templates,scripts,docs,examples}
mkdir -p power-automate
mkdir -p scripts

# 本ガイドから各ファイルをコピー
\end{lstlisting}

\section{ステップ 2: GitHub Secrets 設定}

\subsection{2-1: Personal Access Token (PAT) 生成}

\begin{enumerate}
  \item GitHub Settings → \cmd{Developer settings} → \cmd{Personal access tokens}
  \item \cmd{Generate new token (classic)}
  \item Token 名: \textit{GHCR_DEPLOYMENT_PAT}
  \item スコープ: \checkbox{} \cmd{repo}, \checkbox{} \cmd{write:packages}
  \item \cmd{Generate token}
  \item トークンをコピーして安全に保存
\end{enumerate}

\subsection{2-2: Secrets 設定}

\begin{enumerate}
  \item リポジトリ Settings → \cmd{Secrets and variables} → \cmd{Actions}
  \item \cmd{New repository secret}
  \item 以下を追加:
\end{enumerate}

\begin{table}[h]
\centering
\begin{tabularx}{\textwidth}{|l|X|}
\hline
\textbf{シークレット名} & \textbf{値} \\
\hline
\cmd{PAT\_TOKEN} & 上記で生成した PAT トークン \\
\cmd{POWER\_AUTOMATE\_WEBHOOK} & Power Automate Webhook URL \\
\cmd{GITHUB\_TOKEN} & 自動(リポジトリシークレット) \\
\hline
\end{tabularx}
\caption{GitHub Secrets 設定}
\end{table}

\section{ステップ 3: ローカル開発環境構築}

\subsection{3-1: Python 仮想環境}

\begin{lstlisting}[language=bash,caption=Python 仮想環境構築]
# 仮想環境作成
python3 -m venv venv

# アクティベート
source venv/bin/activate  # Linux/macOS
# または
.\venv\Scripts\activate  # Windows

# 依存関係インストール
pip install --upgrade pip
pip install -r requirements.txt
\end{lstlisting}

\subsection{3-2: Docker ローカルテスト}

\begin{lstlisting}[language=bash,caption=Docker イメージローカルビルド]
# イメージビルド
docker build -t copilot-automation:dev .

# コンテナ実行テスト
mkdir -p input output
echo "テスト画像" > input/test.txt

docker run --rm \
  -v $(pwd)/input:/app/input \
  -v $(pwd)/output:/app/output \
  copilot-automation:dev
\end{lstlisting}

\section{ステップ 4: ワークフロー実行}

\subsection{4-1: GitHub Actions ワークフロー トリガー}

\begin{enumerate}
  \item リポジトリ → \cmd{Actions}
  \item \cmd{Build and Deploy Pipeline} ワークフロー選択
  \item \cmd{Run workflow} → \cmd{Run workflow}
\end{enumerate}

\subsection{4-2: 実行状況確認}

\begin{lstlisting}[language=bash,caption=ワークフロー ログ確認]
# コマンドラインから確認(gh CLI)
gh run list
gh run view <RUN_ID> --log
\end{lstlisting}

\section{ステップ 5: タグリリース}

\subsection{5-1: バージョンタグ作成}

\begin{lstlisting}[language=bash,caption=リリースタグ作成]
# バージョンタグ作成
git tag -a v1.0.0 -m "Release version 1.0.0"

# リモートプッシュ
git push origin v1.0.0

# タグ確認
git tag -l
\end{lstlisting}

\subsection{5-2: GitHub Release 確認}

\begin{enumerate}
  \item リポジトリ → \cmd{Releases}
  \item v1.0.0 をクリック
  \item PDF と Starter Kit ZIP が自動アップロードされていることを確認
\end{enumerate}

\section{ステップ 6: Power Automate 接続}

\subsection{6-1: Webhook URL 生成}

\begin{enumerate}
  \item Power Automate → \cmd{Create} → \cmd{Instant cloud flow}
  \item \cmd{Request} トリガー選択
  \item \cmd{Save}
  \item トリガー内の Webhook URL をコピー
\end{enumerate}

\subsection{6-2: GitHub Secret に登録}

\begin{bash}
gh secret set POWER_AUTOMATE_WEBHOOK --body "https://prod-xx.westus.logic.azure.com:443/..."
\end{bash}

\section{ステップ 7: 検証とテスト}

\subsection{7-1: End-to-End テスト}

\begin{lstlisting}[language=bash,caption=E2E テストシーケンス]
#!/bin/bash

echo "=== End-to-End Test Suite ==="

# 1. LaTeX PDF 生成テスト
echo "[1/5] Testing LaTeX PDF generation..."
cd latex
pdflatex -interaction=nonstopmode guide.tex
pdflatex -interaction=nonstopmode guide.tex
if [ -f guide.pdf ]; then echo "✓ PDF OK"; else echo "✗ PDF NG"; fi

# 2. Docker ビルドテスト
echo "[2/5] Testing Docker build..."
cd ..
docker build -t copilot-automation:test . && echo "✓ Docker OK" || echo "✗ Docker NG"

# 3. ZIP 生成テスト
echo "[3/5] Testing ZIP generation..."
bash scripts/generate-starter-kit.sh && echo "✓ ZIP OK" || echo "✗ ZIP NG"

# 4. Python 依存テスト
echo "[4/5] Testing Python dependencies..."
python -m pip check && echo "✓ Dependencies OK" || echo "✗ Dependencies NG"

# 5. GitHub Secrets テスト
echo "[5/5] Testing GitHub Secrets..."
[ -n "$POWER_AUTOMATE_WEBHOOK" ] && echo "✓ Secrets OK" || echo "✗ Secrets NG"

echo "=== Test Complete ==="
\end{lstlisting}

\subsection{7-2: トラブルシューティング}

\begin{table}[h]
\centering
\begin{tabularx}{\textwidth}{|l|X|l|}
\hline
\textbf{問題} & \textbf{原因} & \textbf{解決策} \\
\hline
LaTeX エラー & フォント不足 & \cmd{tlmgr install cjk} \\
Docker エラー & イメージ大きい & キャッシュクリア \\
Secret 未設定 & リポジトリ設定ミス & Settings 再確認 \\
\hline
\end{tabularx}
\caption{よくあるトラブルと対応}
\end{table}

\section{ステップ 8: 本番環境デプロイ}

\subsection{8-1: ブランチ保護設定}

\begin{enumerate}
  \item Settings → \cmd{Branches} → \cmd{Branch protection rules}
  \item \cmd{Add rule}
  \item Branch name pattern: \textit{main}
  \item \checkbox{} Require status checks
  \item \checkbox{} Require reviews before merging
\end{enumerate}

\subsection{8-2: CI/CD パイプライン検証}

\begin{itemize}
  \item全 Job が成功することを確認
  \item アーティファクトが正常にアップロードされていることを確認
  \item Release ページで PDF と ZIP がダウンロード可能か確認
\end{itemize}

\note{本番環境では必ず develop ブランチで十分なテストを行い、
その後 main ブランチにマージしてください。}

\newpage

% 09-workflow.tex
% ワークフロー図セクション

\chapter{ワークフロー図と実行フロー}

\section{全体ワークフロー図}

\begin{figure}[h]
\centering
\begin{verbatim}
┌─────────────────────────────────────────────────────────────────────┐
│                        開発者の操作                                  │
├─────────────────────────────────────────────────────────────────────┤
│  1. コード修正                                                       │
│  2. git commit & push                                               │
│  3. git tag v1.0.0 (リリース時)                                    │
└────────────────────┬────────────────────────────────────────────────┘
                     │
                     ▼ (Webhook トリガー)
┌─────────────────────────────────────────────────────────────────────┐
│                   GitHub Actions Runner                             │
│                   (ubuntu-latest)                                   │
├──────────────────────────┬──────────────────────────────────────────┤
│  Parallel Job 1          │  Parallel Job 2         │ Parallel Job 3│
│  ─────────────────────   │  ─────────────────────  │ ────────────  │
│  LaTeX PDF Build         │  Docker Image Build     │ Starter Kit   │
│  ├─ Checkout            │  ├─ Setup Buildx       │ ├─ Checkout   │
│  ├─ Cache TeX Live      │  ├─ Login GHCR         │ ├─ Create ZIP │
│  ├─ Install packages    │  ├─ Build image        │ ├─ Upload     │
│  ├─ pdflatex x2         │  ├─ Push to GHCR       │ └─ Output     │
│  └─ Upload artifact     │  └─ Cache layers       │                │
│                          │                        │                │
│  ✓ guide.pdf            │  ✓ ghcr.io/...        │  ✓ kit.zip    │
└──────────────────────────┴──────────────────────────┴────────────────┘
                     │
                     ▼ (Dependency: if success)
┌─────────────────────────────────────────────────────────────────────┐
│              Post-Processing Jobs                                   │
├──────────────────────────┬──────────────────────────────────────────┤
│  Validation              │  Release Upload                          │
│  ─────────────────      │  ───────────────                         │
│  ├─ Download artifacts  │  ├─ Create Release                      │
│  ├─ Verify integrity    │  ├─ Upload PDF + ZIP                    │
│  ├─ Check file sizes    │  └─ Generate release notes              │
│  └─ Log results         │                                          │
│                          │                                          │
│  ✓ Validation OK        │  ✓ Release published                    │
└──────────────────────────┴──────────────────────────────────────────┘
                     │
                     ▼ (Conditional: if tag)
┌─────────────────────────────────────────────────────────────────────┐
│              Trigger External Services                              │
├─────────────────────────────────────────────────────────────────────┤
│  Power Automate Webhook Call                                        │
│  ├─ Send Release Info                                              │
│  ├─ Trigger OCR Flow                                               │
│  └─ Await completion                                               │
│                                                                     │
│  ✓ Webhook sent                                                    │
└────────────────────┬─────────────────────────────────────────────────┘
                     │
                     ▼ (Power Automate)
┌─────────────────────────────────────────────────────────────────────┐
│                  Power Automate Cloud Flow                          │
├──────────────┬────────────────────┬────────────────┬────────────────┤
│ Download ZIP │ Extract Contents   │ OCR Processing │ Save Results   │
│              │                    │                │                │
│ - GitHub API │ - Docker Extract   │ - Tesseract    │ - CSV Convert  │
│ - Filter     │ - Decompress       │ - OCR:jpn+eng  │ - OneDrive     │
│   asset      │ - Read files       │ - Extract text │   Upload       │
│              │                    │                │                │
│ ✓ ZIP ready  │ ✓ Extracted       │ ✓ OCR done    │ ✓ Saved       │
└──────────────┴────────────────────┴────────────────┴────────────────┘
                     │
                     ▼
┌─────────────────────────────────────────────────────────────────────┐
│              Completion & Notification                              │
├─────────────────────────────────────────────────────────────────────┤
│  ✓ Pipeline Complete                                               │
│  📧 Email Notification Sent                                        │
│  📊 Artifacts Available:                                           │
│     - guide.pdf (in GitHub Release)                               │
│     - starter-kit.zip (in GitHub Release)                         │
│     - Docker Image (in GHCR)                                      │
│     - OCR Results CSV (in OneDrive)                               │
└─────────────────────────────────────────────────────────────────────┘
\end{verbatim}
\caption{完全自動化ワークフロー全体図}
\end{figure}

\section{並列実行タイムラインシーケンス}

\begin{figure}[h]
\centering
\begin{verbatim}
Timeline (分)
0     │  5    │  10   │  15   │  20   │  25   │  30
──────┼───────┼───────┼───────┼───────┼───────┼────────

LaTeX │████████████│                 Upload │   ✓
      Build (2x)                      (3m)  │
                                            │
Docker│       ████████████████│             Upload │  ✓
      Build (15m)            Setup          (5m)  │
      (キャッシュなし)
                                            │
Starter│██│                                  ZIP │ ✓
Kit   (1m)                                 (1m)  │
                                            │
─────────────────────────────────────────────────▶ Time
Validation (all jobs)
      └─────────────────────────────────────────┘
      Release Upload
            └─────────────────────────────────────┘
      Power Automate
                  └─────────────────────────────────┘

Overall Duration: ~35 minutes (初回)
               : ~20 minutes (キャッシュ使用時)
\end{verbatim}
\caption{並列実行タイムライン}
\end{figure}

\section{データフロー図}

\begin{figure}[h]
\centering
\begin{verbatim}
Input Sources
┌─────────────┐
│ Repository  │
│ - .tex      │
│ - Dockerfile│
│ - YAML      │
└────┬────────┘
     │
     ▼ (Checkout)
┌─────────────────────────────────────────┐
│     GitHub Actions Environment           │
│     (Workspace)                          │
│     /home/runner/work/repo/repo/         │
└──────────┬────────────────────────────────┘
           │
      ┌────┴────┬────────────────┐
      │          │                │
      ▼          ▼                ▼
┌───────────┐ ┌────────────┐ ┌─────────────┐
│ LaTeX     │ │ Docker     │ │ ZIP Script  │
│ Compiler  │ │ Builder    │ │             │
└────┬──────┘ └────┬───────┘ └────┬────────┘
     │             │              │
     ▼             ▼              ▼
  guide.pdf    Docker Image   starter-kit.zip
     │             │              │
     └─────────────┬──────────────┘
                   │
                   ▼ (Upload)
        GitHub Artifacts Storage
                   │
      ┌────────────┼────────────┐
      │            │            │
      ▼            ▼            ▼
   PDF File    ZIP File    (Ready for Release)
      │            │            │
      └─────────────┴────────────┘
                   │
                   ▼ (Release Create)
          GitHub Release Page
                   │
      ┌────────────┴────────────┐
      │                         │
      ▼ (Download)             ▼ (Download)
   End User PDF          Power Automate
      │                         │
      │                    ┌────┴────────┐
      │                    │             │
      │                    ▼             ▼
      │               Extract Files   OCR Processing
      │                    │             │
      │                ┌───┴─────────────┘
      │                │
      │                ▼
      │           Result CSV
      │                │
      │                ▼
      │          OneDrive Storage
      │
      └─────────────► End User

Data Flow Metrics:
- Total Volume: ~1.5 GB (PDF + Docker layer + ZIP)
- Transfer Time: 2-5 minutes (depending on network)
- Storage Cost: Minimal (GitHub: 1GB free, OneDrive: 1TB free)
\end{verbatim}
\caption{データフロー図}
\end{figure}

\section{エラーハンドリングフロー}

\begin{figure}[h]
\centering
\begin{verbatim}
GitHub Actions Job Execution
         │
         ▼
    ┌─ Success? ─┐
    │            │
   NO            YES
    │             │
    ▼             ▼ (Next Job)
  Error         Continue
  Handler
    │
    ├─ Retry Policy
    │  └─ Max 3 retries
    │
    ├─ Log Error Details
    │  ├─ Job output
    │  ├─ Stack trace
    │  └─ Context info
    │
    ├─ Notify Team
    │  ├─ Email notification
    │  ├─ Slack webhook
    │  └─ GitHub issue creation
    │
    └─ Cleanup
       ├─ Delete temp files
       ├─ Stop Docker containers
       └─ Free resources

Final Status:
  ✓ Success → Deploy
  ✗ Failure → Manual Review Required
  ⏸ Skipped → Not triggered
\end{verbatim}
\caption{エラーハンドリングフロー}
\end{figure}

\section{ユースケース別ワークフロー}

\subsection{ユースケース 1: 定期的なドキュメント更新}

\begin{verbatim}
Tuesday 10:00 UTC
      │
      ▼
Scheduled Trigger (workflow_dispatch)
      │
      ├─ Pull latest changes
      ├─ Build LaTeX PDF
      ├─ Upload to Release
      │
      ✓ Documentation Updated
        (GitHub Release ready for download)
\end{verbatim}

\subsection{ユースケース 2: 新しいバージョンリリース}

\begin{verbatim}
Developer Creates Tag
      │
      ▼ git push origin v2.0.0
      │
GitHub Tag Event Triggered
      │
      ├─ Build All Artifacts (Parallel)
      │  ├─ LaTeX PDF
      │  ├─ Docker Image
      │  ├─ Starter Kit ZIP
      │
      ├─ Create GitHub Release
      │  └─ Auto-generate notes
      │
      ├─ Trigger Power Automate
      │  └─ Start OCR processing
      │
      ✓ Release Published & Processing Started
\end{verbatim}

\subsection{ユースケース 3: 開発ブランチでの CI テスト}

\begin{verbatim}
Developer: git push origin feature/new-feature
      │
      ▼
GitHub PR Event (develop -> main)
      │
      ├─ Run Tests & Validation
      │  ├─ CodeQL scan
      │  ├─ Trivy image scan
      │  ├─ Dependency check
      │
      ├─ Comment Results on PR
      │  └─ Build status badge
      │
      ✓ PR Review Ready
        (Maintainer approval required)
\end{verbatim}

\section{パフォーマンス最適化}

\subsection{キャッシュ戦略}

\begin{table}[h]
\centering
\begin{tabularx}{\textwidth}{|l|X|l|}
\hline
\textbf{キャッシュ対象} & \textbf{キー} & \textbf{効果} \\
\hline
TeX Live packages & texlive-version & 10-15 分短縮 \\
Docker layers & gha mode & 5-8 分短縮 \\
Python packages & pip cache & 3-5 分短縮 \\
\hline
\end{tabularx}
\caption{キャッシュ効果測定}
\end{table}

\subsection{実行時間の比較}

\begin{figure}[h]
\centering
\begin{verbatim}
初回実行(キャッシュなし): 35-40 分
├─ LaTeX: 15 分
├─ Docker: 18 分
└─ Validation: 2-7 分

2 回目以降(キャッシュあり): 15-20 分
├─ LaTeX: 5 分
├─ Docker: 8 分
└─ Validation: 2-7 分

最適化効果: 約 50% 時間短縮
\end{verbatim}
\caption{実行時間比較}
\end{figure}

\note{キャッシュの有効期限は 7 日間です。
定期的にクリアしてメモリを節約することをお勧めします。}

\newpage

% 10-readme.tex
% README 内容セクション

\chapter{README 内容と応用}

\section{README.md の役割}

README.md はプロジェクトの入口となるドキュメントです。
本セクションでは、効果的な README 構成と実装方法を解説します。

\section{README 最適構成}

\begin{lstlisting}[language=markdown,caption=README.md の最適構成テンプレート]
# プロジェクト名

[![Badge1](url)](link)
[![Badge2](url)](link)

## 🎯 概要

1 段落で何ができるかを説明(最大 100 語)

### 主な特徴

- ✓ 特徴 1
- ✓ 特徴 2
- ✓ 特徴 3

## 📋 目次

- [インストール](#インストール)
- [使い方](#使い方)
- [アーキテクチャ](#アーキテクチャ)
- [貢献](#貢献)

## 🚀 クイックスタート

\`\`\`bash
# 最小限のセットアップコマンド
git clone https://github.com/...
cd project
bash setup.sh
\`\`\`

## 📦 インストール

### 前提条件
- Git 2.20+
- Docker 20.10+

### インストール手順

1. リポジトリクローン
2. 環境変数設定
3. 依存関係インストール

### シークレット設定

| Key | Value |
|-----|-------|
| GITHUB_TOKEN | auto |
| PAT_TOKEN | your-pat |

## 💻 使い方

### 基本使用例

\`\`\`bash
# ビルド
docker build -t myapp:latest .

# 実行
docker run myapp:latest
\`\`\`

### 詳細な使用例

[詳細ドキュメント](docs/USAGE.md) を参照

## 🏗️ アーキテクチャ

[図を表示]

## 🔒 セキュリティ

[セキュリティ ポリシー](SECURITY.md)

## 📚 ドキュメント

- [ビルド手順](docs/BUILD.md)
- [デプロイ手順](docs/DEPLOY.md)
- [トラブルシューティング](docs/TROUBLESHOOTING.md)

## 🤝 貢献

プルリクエスト歓迎!
[貢献ガイドライン](CONTRIBUTING.md)

## 📄 ライセンス

[MIT License](LICENSE)

## 👥 お問い合わせ

- 📧 Email: info@example.com
- 🐦 Twitter: @handle
- 💬 Discussions: GitHub Discussions
\end{lstlisting}

\section{バッジの追加方法}

\subsection{一般的なバッジ}

\begin{table}[h]
\centering
\begin{tabularx}{\textwidth}{|l|X|}
\hline
\textbf{バッジ} & \textbf{Markdown} \\
\hline
Build Status & \verb|[![Build](url/badge.svg)](url)| \\
License & \verb|[![MIT](url/badge.svg)](LICENSE)| \\
Python Version & \verb|[![Python 3.11](url/badge.svg)](url)| \\
Docker & \verb|[![Docker](url/badge.svg)](url)| \\
\hline
\end{tabularx}
\caption{一般的なバッジ}
\end{table}

\subsection{バッジ生成ツール}

\begin{itemize}
  \item \textbf{Shields.io}: \url{https://shields.io}
  \item \textbf{GitHub Actions Status}: 自動生成
  \item \textbf{Codecov}: カバレッジバッジ
  \item \textbf{Docker Hub}: イメージメタデータ
\end{itemize}

\section{表の効果的な使用}

\begin{lstlisting}[language=markdown,caption=README 内の表の例]
## 機能比較表

| 機能 | 無料版 | Pro版 | Enterprise |
|------|--------|-------|-----------|
| 基本機能 | ✓ | ✓ | ✓ |
| CI/CD | - | ✓ | ✓ |
| カスタムドメイン | - | ✓ | ✓ |
| SLA保証 | - | - | ✓ |

## 環境別ワークフロー表

| 環境 | 用途 | トリガー | 出力 |
|------|------|---------|------|
| dev | 開発テスト | Push to develop | ログ出力 |
| staging | 本番前テスト | PR to main | Test report |
| production | 本番運用 | Release tag | Deployment |
\end{lstlisting}

\section{コード例の効果的な提示}

\subsection{言語指定}

\begin{lstlisting}[language=markdown,caption=複数言語の code block]
# Python 例
\`\`\`python
import os
print(os.getenv('MY_VAR'))
\`\`\`

# bash 例
\`\`\`bash
export MY_VAR=value
\`\`\`

# Docker コマンド例
\`\`\`docker
FROM python:3.11-slim
RUN pip install requests
\`\`\`
\end{lstlisting}

\section{セクション別 README テンプレート}

\subsection{新規プロジェクト用テンプレート}

\begin{lstlisting}[language=markdown,caption=新規プロジェクト README]
# My Awesome Project

## 概要
このプロジェクトは [説明] を目指しています。

## インストール

\`\`\`bash
git clone ...
cd ...
pip install -r requirements.txt
\`\`\`

## 使い方

\`\`\`bash
python main.py --help
\`\`\`

## ライセンス
MIT

## 貢献
Pull Request 歓迎!
\end{lstlisting}

\subsection{既存プロジェクト更新用テンプレート}

\begin{lstlisting}[language=markdown,caption=更新内容を記載した README]
## 変更履歴

### v2.0.0 (2025-12-10)
- ✨ 新機能 A を追加
- 🐛 バグ B を修正
- ⚡ パフォーマンス C を改善

### v1.5.0 (2025-11-10)
- ...

## マイグレーション ガイド

v1.x から v2.0 への更新方法:

1. 依存関係を更新: \`pip install --upgrade package\`
2. 設定ファイルを移行: [マイグレーション手順](docs/MIGRATE.md)
3. テストを再実行
\end{lstlisting}

\section{よくある質問 (FAQ) セクション}

\begin{lstlisting}[language=markdown,caption=README に FAQ を追加]
## よくある質問

### Q: インストールに失敗した
A: [トラブルシューティング](docs/TROUBLESHOOTING.md) を確認してください。

### Q: Docker イメージはどこにある?
A: [GHCR Registry](https://ghcr.io/your-org/project) で公開しています。

### Q: ライセンスは?
A: [MIT License](LICENSE) です。商用利用可能です。

### Q: 日本語ドキュメントはありますか?
A: [日本語ドキュメント](docs/README_ja.md) を参照してください。
\end{lstlisting}

\section{README チェックリスト}

プロジェクト公開前に確認:

\begin{itemize}
  \item \checkbox{} 明確なプロジェクト説明
  \item \checkbox{} インストール手順
  \item \checkbox{} 使用例(コード付き)
  \item \checkbox{} ライセンス明記
  \item \checkbox{} 貢献ガイドリンク
  \item \checkbox{} トラブルシューティング
  \item \checkbox{} FAQ
  \item \checkbox{} バッジ付き
  \item \checkbox{} リンク確認(全て有効)
  \item \checkbox{} スペリングチェック(特に日本語)
  \item \checkbox{} コード例が実行可能か確認
  \item \checkbox{} ディレクトリ構造が最新か確認
  \item \checkbox{} 前提条件の明記
  \item \checkbox{} 必要なシークレット情報の記載
  \item \checkbox{} サポート窓口情報の記載
\end{itemize}

\section{README メンテナンス}

\subsection{定期更新スケジュール}

\begin{table}[h]
\centering
\begin{tabularx}{\textwidth}{|l|l|}
\hline
\textbf{頻度} & \textbf{更新内容} \\
\hline
毎月 & バージョン情報、リリースノート \\
四半期 & バッジ更新、依存関係の最小バージョン \\
必要に応じて & バグ修正、新機能追加、FAQ 追加 \\
\hline
\end{tabularx}
\caption{README メンテナンススケジュール}
\end{table}

\subsection{バージョン管理}

\begin{lstlisting}[language=bash,caption=README バージョン管理]
# README 内にバージョン情報を含める
---
version: 2.0.0
last-updated: 2025-12-10
maintained: true
---

# README をコミット時に日付付けで記録
git log --oneline -- README.md | head -10
\end{lstlisting}

\section{多言語対応 README}

\begin{lstlisting}[language=markdown,caption=多言語 README サポート]
# プロジェクト名

[日本語](README_ja.md) | [English](README.md) | [中文](README_zh.md)

## English Version

# Project Name

[Content in English]

---

## 日本語版

[日本語コンテンツ]
\end{lstlisting}

\note{GitHub は複数言語の README.md を自動検出します。
言語別ファイル名: README\_en.md, README\_ja.md, etc.}

\section{README 自動生成ツール}

\begin{itemize}
  \item \textbf{readme-md-generator}: npm パッケージ
  \item \textbf{auto-readme}: Python ツール
  \item \textbf{GitHub CLI}: \cmd{gh api} で自動化
\end{itemize}

例:

\begin{lstlisting}[language=bash,caption=readme-md-generator 使用例]
npm install -g readme-md-generator

# 対話的に README 生成
readme
\end{lstlisting}


% ========================================
% 付録(オプション)
% ========================================
\appendix

\chapter{よくある質問(FAQ)}

\section{LaTeX コンパイルができない}
LaTeX の日本語フォントが不足している場合があります。以下のコマンドで TeX Live を更新してください:
\begin{lstlisting}[language=bash,label=lst:texupdate]
tlmgr update --self
tlmgr install cjk xecjk
\end{lstlisting}

\section{Docker ビルドが遅い}
キャッシュレイヤーを活用して高速化できます:
\begin{lstlisting}[language=bash,label=lst:dockerbuild]
docker build --cache-from copilot-automation:latest \
  -t copilot-automation:latest .
\end{lstlisting}

\section{Power Automate トリガーが動作しない}
以下を確認してください:
\begin{enumerate}
  \item Webhook URL が有効か確認
  \item GitHub Secrets で \cmd{POWER\_AUTOMATE\_WEBHOOK} が設定されているか
  \item Power Automate フローが有効か(デザイナーで確認)
\end{enumerate}

% ========================================
% 参考資料
% ========================================
\chapter{参考資料}

\begin{thebibliography}{99}

\bibitem{github-copilot}
GitHub Copilot Documentation.
\textit{https://github.com/features/copilot}

\bibitem{github-actions}
GitHub Actions Documentation.
\textit{https://docs.github.com/en/actions}

\bibitem{docker-docs}
Docker Documentation.
\textit{https://docs.docker.com}

\bibitem{power-automate}
Power Automate Documentation.
\textit{https://powerautomate.microsoft.com}

\bibitem{latex-cjk}
LaTeX CJK Package Documentation.
\textit{https://www.ctan.org/pkg/cjk}

\end{thebibliography}

% セクションファイルを読み込む
\include{sections/overview}
\include{sections/starter-kit}
\include{sections/zip-script}
\include{sections/github-actions}
\include{sections/dockerfile}
\include{sections/power-automate}
\include{sections/security}
\include{sections/deployment-guide}
\include{sections/workflow-diagram}

\end{document}
